% *======================================================================*
%  Cactus Thorn template for ThornGuide documentation
%  Author: Ian Kelley
%  Date: Sun Jun 02, 2002
%  $Header$                                                             
%
%  Thorn documentation in the latex file doc/documentation.tex 
%  will be included in ThornGuides built with the Cactus make system.
%  The scripts employed by the make system automatically include 
%  pages about variables, parameters and scheduling parsed from the 
%  relevent thorn CCL files.
%  
%  This template contains guidelines which help to assure that your     
%  documentation will be correctly added to ThornGuides. More 
%  information is available in the Cactus UsersGuide.
%                                                    
%  Guidelines:
%   - Do not change anything before the line
%       % START CACTUS THORNGUIDE",
%     except for filling in the title, author, date etc. fields.
%        - Each of these fields should only be on ONE line.
%        - Author names should be sparated with a \\ or a comma
%   - You can define your own macros, but they must appear after
%     the START CACTUS THORNGUIDE line, and must not redefine standard 
%     latex commands.
%   - To avoid name clashes with other thorns, 'labels', 'citations', 
%     'references', and 'image' names should conform to the following 
%     convention:          
%       ARRANGEMENT_THORN_LABEL
%     For example, an image wave.eps in the arrangement CactusWave and 
%     thorn WaveToyC should be renamed to CactusWave_WaveToyC_wave.eps
%   - Graphics should only be included using the graphix package. 
%     More specifically, with the "includegraphics" command. Do
%     not specify any graphic file extensions in your .tex file. This 
%     will allow us (later) to create a PDF version of the ThornGuide
%     via pdflatex. |
%   - References should be included with the latex "bibitem" command. 
%   - Use \begin{abstract}...\end{abstract} instead of \abstract{...}
%   - Do not use \appendix, instead include any appendices you need as 
%     standard sections. 
%   - For the benefit of our Perl scripts, and for future extensions, 
%     please use simple latex.     
%
% *======================================================================* 
% 
% Example of including a graphic image:
%    \begin{figure}[ht]
% 	\begin{center}
%    	   \includegraphics[width=6cm]{/home/runner/work/tests/tests/arrangements/Numerical/AEILocalInterp/doc/MyArrangement_MyThorn_MyFigure}
% 	\end{center}
% 	\caption{Illustration of this and that}
% 	\label{MyArrangement_MyThorn_MyLabel}
%    \end{figure}
%
% Example of using a label:
%   \label{MyArrangement_MyThorn_MyLabel}
%
% Example of a citation:
%    \cite{MyArrangement_MyThorn_Author99}
%
% Example of including a reference
%   \bibitem{MyArrangement_MyThorn_Author99}
%   {J. Author, {\em The Title of the Book, Journal, or periodical}, 1 (1999), 
%   1--16. {\tt http://www.nowhere.com/}}
%
% *======================================================================* 

% If you are using CVS use this line to give version information
% $Header$

\documentclass{article}

% Use the Cactus ThornGuide style file
% (Automatically used from Cactus distribution, if you have a 
%  thorn without the Cactus Flesh download this from the Cactus
%  homepage at www.cactuscode.org)
\usepackage{../../../../../doc/latex/cactus}

\newlength{\tableWidth} \newlength{\maxVarWidth} \newlength{\paraWidth} \newlength{\descWidth} \begin{document}

%%%%%%%%%%%%%%%%%%%%%%%%%%%%%%%%%%%%%%%%%%%%%%%%%%%%%%%%%%%%%%%%%%%%%%%%%%%%%%%%

% The author of the documentation
\author{Jonathan Thornburg, incorporating code from Thomas Radke} 

% The title of the document (not necessarily the name of the Thorn)
\title{Thorn Guide for the {\bf AEILocalInterp} Thorn}

% the date your document was last changed, if your document is in CVS, 
% please us:
%    \date{$ $Date$ $}
\date{$ $Date$ $}

\maketitle

% Do not delete next line
% START CACTUS THORNGUIDE

% Add all definitions used in this documentation here 
%   \def\mydef etc
\setlength{\unitlength}{1mm}	% length unit for latex picture environment
\def\csmash#1{\hbox to 0em{\hss{#1}\hss}}

\def\defn#1{{\bf #1}}
\def\thorn#1{{\bf #1}}
\def\arrangement#1{{\bf #1}}

\def\Cplusplus{\hbox{C\hspace{-.05em}\raisebox{.4ex}{\tiny\bf ++}}}
\def\cf{cf.\hbox{}}
\def\Ie{I.e.\hbox{}}
\def\ie{i.e.\hbox{}}
\def\eg{e.g.\hbox{}}

% Add an abstract for this thorn's documentation
\begin{abstract}
This thorn provides the \verb|CCTK_InterpLocalUniform()| API for
processor-local interpolation of N-dimensional data arrays.  The data
arrays (in general there may be many of them) must be defined on a
uniformly spaced grid.

This thorn provides several variants of Lagrange and Hermite
interpolation.  At present the Lagrange interpolation operators support
orders~1 through~6 for 1-D interpolation, and~1 through~4 for 2-D
and 3-D interpolation.  The Hermite interpolation operators support
orders~2, 3, and~4 for 1-D interpolation, and~2 and~3 for 2-D
and 3-D interpolation.

This thorn supports a number of interpolation options, including
parameter-table entries to control the handling of grid boundaries
and out-of-range points, non-contiguous input arrays, taking
derivatives as part of the interpolation, and querying the
Jacobian of the interpolation operation.
\end{abstract}

%%%%%%%%%%%%%%%%%%%%%%%%%%%%%%%%%%%%%%%%%%%%%%%%%%%%%%%%%%%%%%%%%%%%%%%%%%%%%%%%

\section{Introduction}

This interpolator provides:
\begin{itemize}
\item	Lagrange polynomial interpolation of orders 1--6 for 1-D
	interpolation and 1--4 for 2-D and 3-D interpolation.
	This interpolator actually provides two slightly different
	flavors of Lagrange interpolation, \defn{tensor product}
	and \defn{maximum degree}, registered under the operator names
	\begin{verbatim}
   "Lagrange polynomial interpolation (tensor product)"
   "Lagrange polynomial interpolation (maximum degree)"
	\end{verbatim}
	Section~\ref{AEIThorns/AEILocalInterp/sect-multi-dim-interp}
	of this thorn guide explains the two variants in detail, but
	for most purposes the tensor-product variant is
	preferable.  For convenience this is also registered under
	the operator name
	\begin{verbatim}
   "Lagrange polynomial interpolation"
	\end{verbatim}
	and this is probably the best interpolation operator for
	general-purpose use.%%%
\footnote{%%%
	 For backwards compatability this interpolator
	 is also registered under still another operator name,
	 {\tt "generalized polynomial interpolation"}, but this
	 is deprecated and will probably be removed at some point.
	 }%%%
\item	Hermite polynomial interpolation of orders 2-4 for 1-D
	interpolation, and 2--3 for 2-D and 3-D interpolation,
	registered under the operator name
	\begin{verbatim}
   "Hermite polynomial interpolation"
	\end{verbatim}
\end{itemize}
The code allows arbitrarily-shaped interpolation molecules,
but at the moment only hypercube-shaped molecules are implemented.
It would be easy to add additional orders and/or dimensions if desired.

This interpolator supports a number of options specified by a
\defn{parameter table} (a Cactus key-value table, manipulated by the
\verb|Util_Table*()| APIs).  Note that all the interpolator options
described here apply only to the current interpolator call: there is
no visible state kept inside the interpolator from one call to another.

%%%%%%%%%%%%%%%%%%%%%%%%%%%%%%%%%%%%%%%%

\subsection{History}

This interpolator was written by Jonathan Thornburg in winter 2001--2002.
Between then and July~2003 it lived in the
\thorn{LocalInterp} thorn in the \arrangement{CactusBase} arrangement,
but in July 2003 it was moved to the (new)
\thorn{AEILocalInterp} thorn in the \arrangement{AEIThorns} arrangement
so it could stay GPL
(Cactus policies forbid GPL code in the CactusBase arrangement).

%%%%%%%%%%%%%%%%%%%%%%%%%%%%%%%%%%%%%%%%

\subsection{Basic Terminology}
\label{AEIThorns/AEILocalInterp/sect-basic-terminology}

Within Cactus, each interpolator has a character-string name;
this is mapped to a Cactus \defn{interpolator handle} by
\verb|CCTK_InterpHandle()|.  For any given interpolator handle,
there may be a separate interpolator defined for each of the
interpolation APIs (both the processor-local ones provided by this
thorn, and the global ones provided by driver-specific interpolator
thorns such as \thorn{PUGHInterp}).

Terminology for interpolation seems to differ a bit from one author
to another.  Here we refer to the \defn{point-centered} interpolation
of a set of \defn{input arrays} (defining data on a uniformly or
nonuniformly spaced \defn{grid} of \defn{data points}) to a set of
\defn{interpolation points} (specified by a corresponding set of
\defn{interpolation coordinates}), with the results being stored
in a corresponding set of \defn{output arrays}.  Alternatively,
we may refer to \defn{cell-centered} interpolation, using a grid
of \defn{data cells} and a set of \defn{interpolation cells}.
(This latter terminology is common in hydrodynamics interpolators.)

At present all the interpolators do polynomial interpolation, and
allow the interpolation of multiple input arrays (to the same set of
interpolation points) in a single interpolator call, using the basic
algorithm:
\begin{verbatim}
for each interpolation point
{
choose an interpolation molecule position somewhere near the interpolation point

        for each output array
        {
        compute an interpolating polynomial
          which approximates the input data at the molecule points
        output = polynomial(interpolation point)
        }
}
\end{verbatim}
In the future, we may add other interpolators where the choice of
molecule is data-dependent (and thus may vary from one input array to
the next), and/or where the entire input grid is used in each interpolation.

We define the \defn{order} of the interpolation to be the order of
the fitted polynomial.  That is, in our terminology linear interpolation
is order~1, quadratic is order~2, cubic is order~3, etc.  An order~$n$
interpolator thus has $O\big((\Delta x)^{n+1})\big)$ interpolation errors
for generic smooth input data.
Section~\ref{AEIThorns/AEILocalInterp/sect-multi-dim-interp} explains how the
interpolating polynomial is defined for multi-dimensional interpolation.

%%%%%%%%%%%%%%%%%%%%%%%%%%%%%%%%%%%%%%%%

\subsection{The Non-Smoothness of Interpolation Errors}

Because the interpolating polynomial generally changes if
the interpolation point moves from one grid cell to another, unless
something special is done the interpolating function isn't smooth,
\ie{} its 1st~derivative is generically {\em discontinuous\/},
with $O(\Delta x^n)$ jump discontinuities each time the interpolating
polynomial changes.  Correspondingly, the interpolation error is
generically a ``bump function'' which is zero at each grid point
and rises to a local maximum in each grid cell.%%%
\footnote{%%%
	 For a further discussion of the non-smoothness of
	 interpolation errors, see appendix~F of J. Thornburg,
	 \textit{Physical Review D} \textbf{59(10)}, 104007.
	 }%%%
{}  This is the case, for example, for tensor-product Lagrange
polynomial interpolation.%%%
	 

[For maximum-degree Lagrange polynomial interpolation, the
interpolation error may not be zero at the grid points, and the
interpolant itself may not be continuous there.  
Section~\ref{AEIThorns/AEILocalInterp/sect-multi-dim-interp} explains this
in detail.]

This thorn also provides Hermite polynomial interpolation, which
guarantees a smooth ($C^1$) interpolant and (for smooth input data)
interpolation error.  Unfortunately, this comes at the cost of a
larger molecule size than the same-order Lagrange interplator, and
a much larger interpolation error if the interpolation molecules are
off-centered.  (By default, the Hermite interpolator doesn't allow
off-centered molecules, though this can be changed via the
\verb|boundary_off_centering_tolerance[]| and
\verb|excision_off_centering_tolerance[]| parameter-table entries.)

For most purposes Lagrange polynomial interpolation is better;
only use Hermite polynomial interpolation if you need a smooth
interpolant.

%%%%%%%%%%%%%%%%%%%%%%%%%%%%%%%%%%%%%%%%

\subsection{More Terminology}

As described in the Function Reference section of the Cactus User's Guide,
\verb|interp_coords|, \verb|input_arrays|, and \verb|output_arrays| are
actually all pointers to arrays of \verb|void *| pointers, since we
support a number of different Cactus data types.  Internally, the
interpolator casts these \verb|void *| pointers to \verb|CCTK_REAL *|
or whatever the correct Cactus data types are.  But when describing
how the interpolator accesses the various arrays, for simplicity we
usually gloss over this casting, \ie{} we pretend that \verb|interp_coords|,
\verb|input_arrays|, and \verb|output_arrays| are pointers to arrays
of \verb|CCTK_REAL *| pointers.  (This may become clearer once you
read the next example.)

We use \verb|pt|, \verb|in|, and \verb|out| as generic 0-origin integer
subscripts into the arrays of interpolation points, input arrays, and
output arrays respectively.  We use \verb|(i,j,k)| as a generic
\verb|N_dims|-vector of integer subscripts into the input array
\verb|input_arrays[in]|.  (Thus \defn{{\tt (i,j,k)} space} refers to
the grid of data points.)  We usually only write array subscripting
expressions for the 3-D case; the restrictions/generalizations to
other dimensions should be obvious.

For example, for 3-D interpolation, the (x,y,z) coordinates of a typical
interpolation point are given by
\begin{verbatim}
x = interp_coords[0][pt]
y = interp_coords[1][pt]
z = interp_coords[2][pt]
\end{verbatim}
(Notice here that as described above, we've implicitly taken
\verb|interp_coords| to have the C~type\\
\verb|const CCTK_REAL* interp_coords[]|, glossing over the casting
from its actual C~type of\\
\verb|const void* interp_coords[]|.)

We take \verb|axis| to be a 0-origin integer specifying a coordinate
axis (dimension), \ie{} 0~for~$x$, 1~for~$y$, 2~for~$z$, \dots.
We take \verb|ibndry| to be a 0-origin integer specifying the
combination of a coordinate axis (dimension) and a minimum/maximum
``end'' of the grid; these are enumerated in the order
$x_{\min}$, $x_{\max}$, $y_{\min}$, $y_{\max}$, $z_{\min}$,
$z_{\max}$, \dots.

When describing Jacobians and domains of dependence, it's useful to
introduce the notion of \defn{molecule position}, a nominal reference
point for the interpolation molecule in \verb|(i,j,k)| coordinate
space.  (For example, the molecule position might just be the \verb|(i,j,k)|
coordinates of the molecule's center.)  We also introduce
\defn{molecule coordinates} \verb|(mi,mj,jk)|, which are just
\verb|(i,j,k)| coordinates relative to the molecule position.
We use \verb|m| as a generic molecule coordinate.  Thus
(in notation which should be obvious) a generic molecule operation
can be written
\begin{equation}
\verb|output| = \sum_{\tt m} \verb|coeff[posn+m]| \times \verb|input[posn+m]|
\end{equation}
Note that there is no requirement that the output be semantically
located at the position \verb|posn|!  (This may become clearer once
you look at the example in
section~\ref{AEIThorns/AEILocalInterp/sect-Jacobian/fixed-sized-hypercube}.)
However, by convention we do assume that $\verb|m|=0$ is always a valid
\verb|m| coordinate; this simplifies pointer arithmetic.

When describing various entries in the parameter table in,
we use \verb|const| qualifiers on table entries to indicate that the
interpolator treats them as \verb|const| variables/arrays, \ie{} it
promises not to change them.  In contrast, table entries which are
not shown as \verb|const| are considered read-write by the interpolator;
it typically uses them to return outputs back to the caller.

%%%%%%%%%%%%%%%%%%%%%%%%%%%%%%%%%%%%%%%%%%%%%%%%%%%%%%%%%%%%%%%%%%%%%%%%%%%%%%%%

\section{Interpolator Parameters}

The \verb|CCTK_InterpLocalUniform()| interpolation API
(described in detail in the ``Function Reference'' appendix of
the Cactus Users' Guide) includes a \defn{parameter table}
(a Cactus key-value table, manipulated by the \verb|Util_Table*()| APIs)
whose table handle is passed to the interpolator.  This can be used
to specify a number of options to the interpolator; it can also be
used by the interpolator to return additional information back to
the caller.

%%%%%%%%%%%%%%%%%%%%%%%%%%%%%%%%%%%%%%%%

\subsection{Interpolation Order}

The only mandatory parameter for this interpolator is the interpolation
order:
\begin{verbatim}
const CCTK_INT order;
\end{verbatim}
As noted in section~\ref{AEIThorns/AEILocalInterp/sect-basic-terminology},
in our terminology linear interpolation is order~1, quadratic is
order~2, cubic is order~3, etc.

%%%%%%%%%%%%%%%%%%%%%%%%%%%%%%%%%%%%%%%%

\subsection{Molecule Size and Centering}
\label{AEIThorns/AEILocalInterp/sect-molecule-size+centering}

If no grid boundaries or excised points are nearby, the interpolator
centers the molecules around the interpolation point as much as possible.
Table~\ref{AEIThorns/AEILocalInterp/tab-molecule-size+centering} gives the molecule
size and details of the centering policy for each interpolation operator
and order.

%%%%%%%%%%%%%%%%%%%%%%%%%%%%%%%%%%%%%%%%

\begin{table}

\def\Linterval{\lower0.60ex\hbox{\hskip-0.20em$\big[$}}
\def\Rinterval{\lower0.60ex\hbox{\hskip-0.25em$\big)$}}

\def\TwoPointMolecule{%%%
\raise0.5ex\hbox{%%%
\begin{picture}(0,0)
\put(-4,0){\circle*{1.5}}
\put(-4,0){\Linterval}
\put(4,0){\Rinterval}
\put(4,0){\circle*{1.5}}
\end{picture}%%%
}}%%%

\def\ThreePointMolecule{%%%
\raise0.5ex\hbox{%%%
\begin{picture}(0,0)
\put(-8,0){\circle*{1.5}}
\put(-4,0){\Linterval}
\put(0,0){\circle*{1.5}}
\put(4,0){\Rinterval}
\put(8,0){\circle*{1.5}}
\end{picture}%%%
}}%%%

\def\FourPointMolecule{%%%
\raise0.5ex\hbox{%%%
\begin{picture}(0,0)
\put(-12,0){\circle*{1.5}}
\put(-4,0){\circle*{1.5}}
\put(-4,0){\Linterval}
\put(4,0){\Rinterval}
\put(4,0){\circle*{1.5}}
\put(12,0){\circle*{1.5}}
\end{picture}%%%
}}%%%

\def\FivePointMolecule{%%%
\raise0.5ex\hbox{%%%
\begin{picture}(0,0)
\put(-16,0){\circle*{1.5}}
\put(-8,0){\circle*{1.5}}
\put(-4,0){\Linterval}
\put(0,0){\circle*{1.5}}
\put(4,0){\Rinterval}
\put(8,0){\circle*{1.5}}
\put(16,0){\circle*{1.5}}
\end{picture}%%%
}}%%%

\def\SixPointMolecule{%%%
\raise0.5ex\hbox{%%%
\begin{picture}(0,0)
\put(-20,0){\circle*{1.5}}
\put(-12,0){\circle*{1.5}}
\put(-4,0){\circle*{1.5}}
\put(-4,0){\Linterval}
\put(4,0){\Rinterval}
\put(4,0){\circle*{1.5}}
\put(12,0){\circle*{1.5}}
\put(20,0){\circle*{1.5}}
\end{picture}%%%
}}%%%

\def\SevenPointMolecule{%%%
\raise0.5ex\hbox{%%%
\begin{picture}(0,0)
\put(-24,0){\circle*{1.5}}
\put(-16,0){\circle*{1.5}}
\put(-8,0){\circle*{1.5}}
\put(-4,0){\Linterval}
\put(0,0){\circle*{1.5}}
\put(4,0){\Rinterval}
\put(8,0){\circle*{1.5}}
\put(16,0){\circle*{1.5}}
\put(24,0){\circle*{1.5}}
\end{picture}%%%
}}%%%

\renewcommand{\arraystretch}{1.5}
\begin{flushleft}
\begin{tabular}{@{}lcc@{\quad\hspace{25mm}}c@{\hspace{25mm}}}
						&	& Molecule
	&								\\[-1ex]
Interpolation Operator				& Order	& Size
	& \csmash{Molecule Centering Policy}				\\
\hline %-------------------------------------------------------
\verb|"Lagrange polynomial interpolation"|	& 1	& 2-point
	& \TwoPointMolecule						\\
\verb|"Lagrange polynomial interpolation"|	& 2	& 3-point
	& \ThreePointMolecule						\\
\verb|"Lagrange polynomial interpolation"|	& 3	& 4-point
	& \FourPointMolecule						\\
\verb|"Lagrange polynomial interpolation"|	& 4	& 5-point
	& \FivePointMolecule						\\
\verb|"Lagrange polynomial interpolation"|	& 5	& 6-point
	& \SixPointMolecule						\\
\verb|"Lagrange polynomial interpolation"|	& 6	& 7-point
	& \SevenPointMolecule						\\
\hline %-------------------------------------------------------
\verb|"Hermite polynomial interpolation"|	& 2	& 4-point
	& \FourPointMolecule						\\
\verb|"Hermite polynomial interpolation"|	& 3	& 6-point
	& \SixPointMolecule						\\
\verb|"Hermite polynomial interpolation"|	& 4	& 6-point
	& \SixPointMolecule						\\
\hline %-------------------------------------------------------
\end{tabular}
\end{flushleft}
\caption[Molecule Sizes]
	{
	This table gives the molecule size and centering for each
	interpolation operator and order.  $\bullet$~shows the input
	grid points, and $[ \quad )$~shows the interval of interpolation
	points (relative to the grid) for which the interpolator
	will use the molecule shown.
	For example, with grid points at integer coordinates, if
	the the interpolator is using 5-point molecules, it will
	use a molecule containing the grid points $\{ -2, -1, 0, 1, 2 \}$
	for all interpolation points in the range $[-0.5, 0.5)$.
	}
\label{AEIThorns/AEILocalInterp/tab-molecule-size+centering}
\end{table}

%%%%%%%%%%%%%%%%%%%%%%%%%%%%%%%%%%%%%%%%

\subsection{Handling of Grid Boundaries}
\label{AEIThorns/AEILocalInterp/sect-grid-boundaries}

If an interpolation point is too near a grid boundary and/or an
excised region the interpolator can either off-center the interpolation
molecule, or refuse to interpolate that point
(returning a\\
\verb|CCTK_INTERP_ERROR_POINT_OUTSIDE|\,%%%
\footnote{%%%
	 For backwards compatability the error code
	 {\tt CCTK\_INTERP\_ERROR\_POINT\_X\_RANGE}
	 is also defined; it's a synonym for
	 {\tt CCTK\_INTERP\_ERROR\_POINT\_OUTSIDE}.
	 }%%%
{} or \verb|CCTK_INTERP_ERROR_POINT_EXCISED| error code, and possibly
also printing a warning message).  This behavior is controlled by the
(optional) parameter-table entries
\begin{verbatim}
const CCTK_INT N_boundary_points_to_omit[2*N_dims]
const CCTK_REAL boundary_off_centering_tolerance[2*N_dims]
const CCTK_REAL boundary_extrapolation_tolerance[2*N_dims]
const CCTK_REAL excision_off_centering_tolerance[2*N_dims] # not implemented yet
const CCTK_REAL excision_extrapolation_tolerance[2*N_dims] # not implemented yet
\end{verbatim}
The elements of these arrays are matched up with the grid boundaries
in the order\\
$[x_{\min}, x_{\max}, y_{\min}, y_{\max}, z_{\min}, z_{\max}, \dots]$.
(For \verb|excision_*_tolerance[]| the minimum/maximum are interpreted
as the corresponding coordinate decreasing/increasing when moving
out of the active region of the grid into the excised region.)
We use \verb|ibndry| as a generic 0-origin index into these arrays.

%%%%%%%%%%%%%%%%%%%%

\subsubsection{Omitting Boundary Points}

For any given grid boundary \verb|ibndry|, the interpolator doesn't use
input data from the\\
\verb|N_boundary_points_to_omit[ibndry]| grid points closest to the
boundary.  In other words, these points are effectively omitted from
the input grid.  For example:
\begin{itemize}
\item	Setting this parameter to an array of all 0s means the
	interpolator may use data from all grid points.
	This is the default if this parameter isn't specified.
\item	Setting this parameter to an array of all 1s means the
	interpolator may not use input data from the outermost
	row/plane of grid points on any face.
\end{itemize}

More generally, this option may be useful for controlling the extent
to which the interpolator uses input data from ghost and/or symmetry
zones.

%%%%%%%%%%%%%%%%%%%%

\subsubsection{Off-Centering Molecules near Grid Boundaries
	       and/or Excision Regions}

The (optional) parameter-table entries
\begin{verbatim}
const CCTK_REAL boundary_off_centering_tolerance[2*N_dims]
const CCTK_REAL boundary_extrapolation_tolerance[2*N_dims]
const CCTK_REAL excision_off_centering_tolerance[2*N_dims] # not implemented yet
const CCTK_REAL excision_extrapolation_tolerance[2*N_dims] # not implemented yet
\end{verbatim}
control the local interpolator's willingness to off-center
(change away from the default centering) its interpolation molecules
near boundaries and excised points.

To describe how these parameters work, it's useful to define two
regions in the data coordinate space:
\begin{description}
\item[The \defn{valid-data region}]
	is the entire grid minus any ommited-on-boundary
	or excised points; to make this a region we take the
	Lego-block bounding box of the non-excised grid points.
\item[The \defn{default-centering region}]
	is that region in interpolation-point space where the
	interpolator can use the default molecule centering
	(described in table~\ref{AEIThorns/AEILocalInterp/tab-molecule-size+centering}),
	\ie{} where the default-centering molecules require data
	only from the data-valid region.
\end{description}
Figure~\ref{AEIThorns/AEILocalInterp/fig-valid-data+default-centering} shows an
example of these regions.

%%%%%%%%%%%%%%%%%%%%%%%%%%%%%%%%%%%%%%%%

\begin{figure}

% "X" to mark excised point
\def\ExcisedPoint{%%%
\begin{picture}(0,0)
\put(-2,-2){\line(1,1){4}}
\put(2,-2){\line(-1,1){4}}
\end{picture}%%%
}%%%

\def\Linterval{\lower0.60ex\hbox{\hskip-0.20em$\big[$}}

\def\xmin{\begin{picture}(0,0)\put(4,0){\vector(-1,0){4}}\end{picture}}
\def\xmax{\begin{picture}(0,0)\put(-4,0){\vector(1,0){4}}\end{picture}}
\def\ymin{\begin{picture}(0,0)\put(0,4){\vector(0,-1){4}}\end{picture}}
\def\ymax{\begin{picture}(0,0)\put(0,-4){\vector(0,1){4}}\end{picture}}

\begin{center}
\begin{picture}(150,150)
%%% grid
\multiput(0,0)(10,0){16}{\multiput(0,0)(0,10){16}{\circle*{1.5}}}
%%% excised points
\multiput(70,60)(10,0){2}{\ExcisedPoint}
\multiput(60,70)(10,0){4}{\ExcisedPoint}
\multiput(60,80)(10,0){4}{\ExcisedPoint}
\multiput(70,90)(10,0){2}{\ExcisedPoint}
%%% outer-boundary valid-data region
\multiput(0,0)(2,0){75}{\line(1,0){1}}
\multiput(150,0)(0,2){75}{\line(0,1){1}}
\multiput(150,150)(-2,0){75}{\line(-1,0){1}}
\multiput(0,150)(0,-2){75}{\line(0,-1){1}}
% ... and movement directions
\put(75,10){\ymin}
\put(140,75){\xmax}
\put(75,140){\ymax}
\put(10,75){\xmin}
%%% outer-boundary default-centering regions...
% ... for 3-point molecules
\put(5,5){\line(1,0){140}}
\put(145,5){\line(0,1){140}}
\put(145,145){\line(-1,0){140}}
\put(5,145){\line(0,-1){140}}
% ... for 4-point molecules
\put(10,10){\line(1,0){130}}
\put(140,10){\line(0,1){130}}
\put(140,140){\line(-1,0){130}}
\put(10,140){\line(0,-1){130}}
%%% excised-point valid-data region
\multiput(60,60)(0,-2){5}{\line(0,-1){1}}
\multiput(60,50)(2,0){15}{\line(1,0){1}}
\multiput(90,50)(0,2){5}{\line(0,1){1}}
\multiput(90,60)(2,0){5}{\line(1,0){1}}
\multiput(100,60)(0,2){15}{\line(0,1){1}}
\multiput(100,90)(-2,0){5}{\line(-1,0){1}}
\multiput(90,90)(0,2){5}{\line(0,1){1}}
\multiput(90,100)(-2,0){15}{\line(-1,0){1}}
\multiput(60,100)(0,-2){5}{\line(0,-1){1}}
\multiput(60,90)(-2,0){5}{\line(-1,0){1}}
\multiput(50,90)(0,-2){15}{\line(0,-1){1}}
\multiput(50,60)(2,0){5}{\line(1,0){1}}
%%% excised-point default-centering regions...
% ... for 3-point molecules
\put(55,55){\line(0,-1){10}}
\put(55,45){\line(1,0){40}}
\put(95,45){\line(0,1){10}}
\put(95,55){\line(1,0){10}}
\put(105,55){\line(0,1){40}}
\put(105,95){\line(-1,0){10}}
\put(95,95){\line(0,1){10}}
\put(95,105){\line(-1,0){40}}
\put(55,105){\line(0,-1){10}}
\put(55,95){\line(-1,0){10}}
\put(45,95){\line(0,-1){40}}
\put(45,55){\line(1,0){10}}
% ... for 4-point molecules
\put(50,50){\line(0,-1){10}}
\put(50,40){\line(1,0){50}}
\put(100,40){\line(0,1){10}}
\put(100,50){\line(1,0){10}}
\put(110,50){\line(0,1){50}}
\put(110,100){\line(-1,0){10}}
\put(100,100){\line(0,1){10}}
\put(100,110){\line(-1,0){50}}
\put(50,110){\line(0,-1){10}}
\put(50,100){\line(-1,0){10}}
\put(40,100){\line(0,-1){50}}
\put(40,50){\line(1,0){10}}
% ... and movement directions for 4-point molecules
\put(50,45){\xmax}
\put(75,40){\ymax}
\put(100,45){\xmin}
\put(105,50){\ymax}
\put(110,75){\xmin}
\put(105,100){\ymin}
\put(100,105){\xmin}
\put(75,110){\ymin}
\put(50,105){\xmax}
\put(45,100){\ymin}
\put(40,75){\xmax}
\put(45,50){\ymax}
\end{picture}
\end{center}

\caption[Example of valid-data and default-centering regions]
	{
	This figure shows an example of the valid-data and
	default-centering regions for a 2-D grid.
	$\times$~marks the excised points, and the dashed lines
	show the boundaries of the valid-data region.
	The solid lines show the boundaries of the default-centering
	regions for 3-point and 4-point molecules.  The arrows on the
	default-centering region for 4-point molecules show which
	elements of the \hbox{\tt boundary\_*\_tolerance[ibndry]}
	and \hbox{\tt excision\_*\_tolerance[ibndry]} arrays apply
	to each line segment of the region boundary:\\
\centerline{%%%
\renewcommand{\arraystretch}{1.25}
\begin{tabular}{c@{\qquad}r}
$\raise0.50ex\hbox{\xmin \hskip4mm}$	& $x_{\min}$ ({\tt ibndry = 0})	\\
$\raise0.50ex\hbox{\hskip4mm \xmax}$	& $x_{\max}$ ({\tt ibndry = 1})	\\
$\lower1mm\hbox{\,\ymin\,}$		& $y_{\min}$ ({\tt ibndry = 2})	\\
$\raise3mm\hbox{\,\ymax\,}$		& $y_{\max}$ ({\tt ibndry = 3})	%%%\\
\end{tabular}%%%
}
	}
\label{AEIThorns/AEILocalInterp/fig-valid-data+default-centering}
\end{figure}

%%%%%%%%%%%%%%%%%%%%%%%%%%%%%%%%%%%%%%%%

Near a boundary or excised regions, the interpolator will off-center
its molecules (if and) only if the interpolation point is both
\begin{itemize}
\item	within ($\le$) \verb|*_off_centering_tolerance[ibndry]|
	grid spacings of the default-centering region, {\bf and}
\item	within ($\le$) \verb|*_extrapolation_tolerance[ibndry]|
	grid spacings of the valid-data region.
\end{itemize}
where we use ``\verb|*|'' to denote \verb|boundary| or \verb|excision|
as appropriate.

There are four cases for these parameters:
\begin{description}
\item[\mathversion{bold}
      $\text{\tt *\!\_off\_centering\_tolerance[ibdnry]} = 0.0$ and
      $\text{\tt *\!\_extrapolation\_tolerance[ibndry]} = 0.0$:]\mbox{}\\
	No off-centering is allowed: the interpolator reports an
	error (\verb|CCTK_ERROR_INTERP_POINT_OUTSIDE| or
	\verb|CCTK_ERROR_INTERP_POINT_EXCISED| return code
	as appropriate) if any interpolation point is outside
	the default-centering region.
\item[\mathversion{bold}
      $\text{\tt *\!\_off\_centering\_tolerance[ibdnry]} > 0.0$ and
      $\text{\tt *\!\_extrapolation\_tolerance[ibndry]} = 0.0$:]\mbox{}\\
	The interpolator allows interpolation points to be up to ($\le$)
	\verb|*_off_centering_tolerance[ibndry]| grid spacings outside
	the default-centering region in this direction.
	If an interpolation point is beyond this limit in any
	direction, then the interpolator reports an error.
\item[\mathversion{bold}
      $\text{\tt *\!\_off\_centering\_tolerance[ibdnry]} = \infty$%%%
\footnotemark{}%%%
{}    and
      $\text{\tt *\!\_extrapolation\_tolerance[ibndry]} > 0.0$:]\mbox{}\\
\footnotetext{%%%
	     In practice $999.0$ is a conventional value here.
	     The actual value doesn't matter so long as it's
	     larger than any possible molecule radius.
	     }%%%
	The interpolator allows interpolation points to be up to ($\le$)
	\verb|*_extrapolation_tolerance[ibndry]| grid spacings outside
	the valid-data region in this direction.
	If an interpolation point is beyond this limit in any
	direction, then the interpolator reports an error.
\item[\mathversion{bold}
      $\text{\tt *\!\_off\_centering\_tolerance[ibdnry]} = 0.0$ and
      $\text{\tt *\!\_extrapolation\_tolerance[ibndry]} > 0.0$:]\mbox{}\\
	In practice the default-centering region is always a
	(normally proper) subset of the valid-data region, so this
	case is nonsensical: the positive value for
	\verb|*_extrapolation_tolerance[ibndry]| has no effect
	because of the $\verb|*_off_centering_tolerance[ibndry]| = 0.0$
	setting.  The interpolator gives a warning for this case.
\end{description}

If any of these table entries aren't specified, the defaults are
\begin{verbatim}
boundary_off_centering_tolerance[ibndry] = 999.0   # effectively infinity
boundary_extrapolation_tolerance[ibndry] = 1.0e-10
\end{verbatim}
In other words, the interpolation points may be anywhere within the
valid-data region or up to $10^{-10}$ grid spacing outside it.  (The
$10^{-10}$ ``fudge factor'' helps to avoid spurious errors in case
floating-point roundoff moves an interpolation point which was supposed
to be just on the boundary of the valid-data region, slightly outside it.)

[In the near future these defaults may be changed:  For Lagrange
polynomial interpolation the new defaults would be
\begin{verbatim}
boundary_off_centering_tolerance[ibndry] = 999.0   # effectively infinity
boundary_extrapolation_tolerance[ibndry] = 1.0e-10
\end{verbatim}
while for Hermite polynomial interpolation they would be
\begin{verbatim}
boundary_off_centering_tolerance[ibndry] = 1.0e-10
boundary_extrapolation_tolerance[ibndry] = 0.0
\end{verbatim}
This would leave Lagrange interpolation unchanged, while for Hermite
interpolation the defaults would forbid any significant off-centering
of the interpolation molecules.]

%%%%%%%%%%%%%%%%%%%%

\subsubsection{Suppressing Warning Messages about Off-Centering}

By default, \thorn{AEILocalInterp} prints a Cactus level~1 warning
message for each interpolation point which causes it to return a
\verb|CCTK_INTERP_ERROR_POINT_OUTSIDE| or
\verb|CCTK_INTERP_ERROR_POINT_EXCISED| error code.  If the key
\verb|suppress_warnings| is present in the parameter table
(it may have any datatype and value), then these messages are
not printed.

%%%%%%%%%%%%%%%%%%%%%%%%%%%%%%%%%%%%%%%%

\subsection{Per-Point Status Reporting}

By default, an interpolator just returns a single result, either
0 for success or some (negative) error code.  If there are multiple
interpolation points causing errors, an interpolator should return
0 for success, or the error code for the first point in error
(the one with the smallest \verb|pt|).

However, sometimes you would like to know the status of the interpolation
for {\em each\/} point.  The following (optional) parameter-table entries
may be used to request this, and to query if the interpolator supports this:
\begin{verbatim}
/*
 * To query whether the interpolator supports per-point status, set this
 * to a NULL pointer.  To actually request per-point status, set this to
 * a non-NULL pointer pointing to a buffer of  N_interp_points  CCTK_INTs
 * into which the interpolator should store the per-point status.  The status
 * for point  pt  is defined to be the result which CCTK_InterpLocalUnifom()
 * would return if that were the only point being interpolated.
 */
CCTK_POINTER per_point_status;
\end{verbatim}

If the interpolator supports returning per-point status, and if the key
\verb|per_point_status| is present in the parameter table, then the
interpolator will set
\begin{verbatim}
CCTK_INT error_point_status;
\end{verbatim}
in the parameter table.  If $\verb|per_point_status| = \verb|NULL|$,
the interpolator will set \verb|error_point_status| to~0.  Otherwise,
the interpolator will set \verb|error_point_status| to the negative of
the number of points for which the per-point status is non-zero.

%%%%%%%%%%%%%%%%%%%%%%%%%%%%%%%%%%%%%%%%

\subsection{Multi-Dimensional Interpolation}
\label{AEIThorns/AEILocalInterp/sect-multi-dim-interp}

In multiple dimensions, there are two plausible definitions for the
generic interpolating polynomial of a given order (degree) $n$.  For
convenience I'll describe these for the 2-D case, but the generalization
to any number of dimensions should be obvious:

\begin{itemize}
\item	The generic ``\defn{tensor product}'' polynomial of degree $n$ is
	\begin{equation}
	f(x,y) = \sum_{{0 \le i \le n} \atop {0 \le j \le n}} a_{ij} x^i y^j
			\label{AEIThorns/AEILocalInterp/eqn-polynomial-2d-TP}
	\end{equation}
\item	The generic ``\defn{maximum degree}'' polynomial of degree $n$ is
	\begin{equation}
	g(x,y) = \sum_{0 \le i+j \le n} a_{ij} x^i y^j
			\label{AEIThorns/AEILocalInterp/eqn-polynomial-2d-MD}
	\end{equation}
\end{itemize}

Because it has $(n+1)^2$ coefficients, the tensor-product polynomial $f$
can (and does) pass through all the $(n+1)^2$ input data points in a
square molecule of size $n+1$.  This implies that the interpolation
error vanishes at the input grid points, and that the overall
interpolating function is continuous (up to floating-point roundoff
errors).  However, $f$ does have the slightly peculiar property of
having terms up to $x^n y^n$ despite being of ``degree'' $n$.
(For example, the ``linear'' interpolant for $n=1$ would have $xy$ terms,
even though those are formally quadratic in the independent variables
$x$ and $y$.)

In contrast, the maximum-degree polynomial $g$ has only
$\frac{1}{2} (n+1)(n+2)$ coefficients, so for generic input data
(\ie{} input data which isn't actually sampled from a polynomial of
the form~$(\ref{AEIThorns/AEILocalInterp/eqn-polynomial-2d-MD})$) $g$ can't pass
through all the $(n+1)^2$ input data points in a square molecule of
size $n+1$.  The interpolator actually does a least-squares fit of
the polymomial $g$ to the input data in the molecule, so in general
the molecule won't pass through {\em any\/} of the data points!
Moreover, each time the interpolation point crosses a grid square
(for odd~$n$) or the center lines of a grid square (for even~$n$),
the set of points used in the molecule changes 
(\cf{}~section~\ref{AEIThorns/AEILocalInterp/sect-molecule-size+centering}),
so the interpolant generally has a jump discontinuity!
For these reasons, the tensor-product
choice~$(\ref{AEIThorns/AEILocalInterp/eqn-polynomial-2d-TP})$ is generally
preferable.

%%%%%%%%%%%%%%%%%%%%%%%%%%%%%%%%%%%%%%%%

\subsection{Molecule Family}
\label{AEIThorns/AEILocalInterp/sect-molecule-family}

An interpolator may support different families/shapes of interpolation
molecules.  Hypercube-shaped molecules are the simplest and most common
case, but one could also imagine (say) octagon-shaped molecules in 2-D,
or some generalization of this in higher numbers of dimensions.

The (optional) parameter
\begin{verbatim}
/* set with Util_TableSetString() */
const char molecule_family[];
\end{verbatim}
may be used to set or query the molecule family.

If this key is present in the parameter table, the interpolator sets
the molecule family/shape based on the value specified.
If this key {\em isn't\/} present in the parameter table, then the
interpolator sets it to the molecule family being used.

At present only hypercubed-shaped molecules are implemented; these
are specified by \verb|molecule_family| set to \verb|"cube"|.

%%%%%%%%%%%%%%%%%%%%%%%%%%%%%%%%%%%%%%%%

\subsection{Non-Contiguous Input Arrays}
\label{AEIThorns/AEILocalInterp/sect-non-contiguous-inputs}

Sometimes the input ``arrays'' used by the interpolator might not
be contiguous in memory.  For example, we might want to do 2-D interpolation
within a plane of a 3-D grid array, but the plane might or might not
be contiguous in memory.  (Another example would be that the input
arrays might be members of a compact group.)

The following (optional) parameter-table entries specify non-contiguous
input arrays:
\begin{verbatim}
const CCTK_INT input_array_offsets[N_input_arrays];

/* the next 3 table entries are shared by all input arrays */
const CCTK_INT input_array_strides       [N_dims];
const CCTK_INT input_array_min_subscripts[N_dims];
const CCTK_INT input_array_max_subscripts[N_dims];
\end{verbatim}

In general, the interpolator accesses the input using the generic
subscripting expression
\begin{verbatim}
input_array[in][offset + i*stride_i + j*stride_j + k*stride_k]
\end{verbatim}
where
\begin{verbatim}
offset = input_array_offsets[in]
(stride_i,stride_j,stride_k) = input_array_strides[]
\end{verbatim}
and where \verb|(i,j,k)| run from \verb|input_array_min_subscripts[]|
to \verb|input_array_max_subscripts[]| inclusive
(n.b.~this is an {\em inclusive\/} range, \ie{} 
$\verb|min| \le \verb|(i,j,k)| \le \verb|max|$).

The defaults are that each input array is contiguous in memory,
\ie{} \verb|input_array_offsets[]| = 0,
\verb|stride| determined from \verb|input_array_dims[]|
              in the usual Fortran manner,
\verb|input_array_min_subscripts[]| = all 0s, and
\verb|input_array_max_subscripts[]| = \verb|input_array_dims[]|-1.
If the stride and max subscript are both specified explicitly, then the
explicit \verb|input_array_dims[]| argument to
\verb|CCTK_InterpLocalUniform()| is ignored.

%%%%%%%%%%%%%%%%%%%%%%%%%%%%%%%%%%%%%%%%

\subsection{Derivatives}
\label{AEIThorns/AEILocalInterp/sect-derivatives}

If we view the input data as being samples of a smooth function,
then instead of estimating values of that function at the interpolation
points, the interpolator can instead or additionally estimate values
of various derivatives of that function at the interpolation points.
(We don't currently implement it, but one could also imagine
interpolating more general molecule operations such as Laplacians.)

The following (optional) parameter-table entries are used to specify
this:
\begin{verbatim}
const CCTK_INT operand_indices[N_output_arrays];
const CCTK_INT operation_codes[N_output_arrays];
\end{verbatim}
The semantics here are that
\begin{verbatim}
output_array[out] = op(input_array[operand_indices[out]])
\end{verbatim}
where \verb|op| is an operator specified by the \verb|operation_codes[out]|
value as described below.

Note that \verb|operand_indices[]| doesn't directly name the inputs,
but rather gives their (0-origin) subscripts in the list of inputs.
This allows for a more efficient implementation in the (common) case
where some of the input arrays have many different operations applied
to them.  (It's most efficient to group all operations on a given
input array together in the \verb|operand_indices| and
\verb|operation_codes| arrays, as in the example in
section~\ref{AEIThorns/AEILocalInterp/sect-example-derivatives}.)

Negative \verb|operation_codes[out]| values  are reserved for future
use.  An \verb|operation_codes[out]| value which is $\ge 0$ is taken
as a decimal integer encoding a coordinate partial derivative: each
decimal digit means to take the coordinate partial derivative along
that (1-origin) axis; the order of the digits in a number is ignored.
Table~\ref{AEIThorns/AEILocalInterp/tab-derivative-codes} summarizes these resulting
derivative codes.

%%%%%%%%%%%%%%%%%%%%%%%%%%%%%%%%%%%%%%%%

\begin{table}
\begin{center}
\begin{tabular}{cl}
\verb|operation_codes[out]|
	& What it means							\\
\hline %---------------------------------------------------------------
0	& interpolate the input array itself (no derivative)		\\[1ex]
1	& interpolate $\partial \big/ \partial x^1$ of the input array	\\
2	& interpolate $\partial \big/ \partial x^2$ of the input array	\\
3	& interpolate $\partial \big/ \partial x^3$ of the input array	\\[1ex]
12 or 21& interpolate $\partial^2 \big/ \partial x^1 \partial x^2$
	  of the input array						\\
13 or 31& interpolate $\partial^2 \big/ \partial x^1 \partial x^3$
	  of the input array						\\
23 or 32& interpolate $\partial^2 \big/ \partial x^2 \partial x^3$
	  of the input array						\\
11	& interpolate $\partial^2 \big/ \partial (x^1)^2$ of the input array \\
22	& interpolate $\partial^2 \big/ \partial (x^2)^2$ of the input array \\
33	& interpolate $\partial^2 \big/ \partial (x^3)^2$ of the input array \\
\hline %---------------------------------------------------------------
\end{tabular}
\end{center}
\caption[Derivative Codes]
	{
	This table gives the codes in {\tt operation\_codes[out]}
	for each possible 1st or 2nd~derivative in 3-D; for 1-D or
	2-D codes referring to nonexistent coordinates are invalid.
	}
\label{AEIThorns/AEILocalInterp/tab-derivative-codes}
\end{table}

%%%%%%%%%%%%%%%%%%%%%%%%%%%%%%%%%%%%%%%%

At present we do {\em not\/} have any \verb|#define|s for the
operation codes in the Cactus header files.  However, you can avoid
most of the software-engineering problems of having ``magic numbers''
for the operation codes, by using the macro
\begin{verbatim}
#define DERIV(op)   op
\end{verbatim}
to mark all such \verb|operation_codes[]| values in your code.
There's an example of this in
section~\ref{AEIThorns/AEILocalInterp/sect-example-derivatives}.

%%%%%%%%%%%%%%%%%%%%%%%%%%%%%%%%%%%%%%%%

\subsection{Jacobian and Domain of Dependence}
\label{AEIThorns/AEILocalInterp/sect-Jacobian}

The Jacobian of the interpolator is defined as
\begin{equation}
\frac{\partial \hbox{\tt output\_array[out][pt]}}
     {\partial \hbox{\tt input\_array[in][(i,j,k)]}}
				\label{AEIThorns/AEILocalInterp/eqn-Jacobian}
\end{equation}
We may want to know the Jacobian itself, and/or ``just'' the set of
\verb|(i,j,k)| for which this is nonzero (\ie{} the Jacobian's sparsity
structure, or equivalently the domain of dependence of the output result,
or equivalently the interpolation molecule size and shape) for a given
\verb|out|, \verb|in|, and \verb|pt|.%%%
\footnote{%%%
	 For something like a spline interpolator the Jacobian
	 is generally nonzero for all {\tt (i,j,k)}, but
	 exponentially small for most of them; in this case
	 the Jacobian-query API would probably specify a cutoff
	 to return an approximate Jacobian with reasonable sparsity.
	 }%%%

The complexity of doing this depends (strongly!) on the structure
of the Jacobian, and in particular on the answers to the following
questions:
\begin{itemize}
\item	What molecule family is being used?
\item	Does the interpolation molecule size and/or shape depend
	on where the interpolation points are in the grid?%%%
\footnote{%%%
	 We can always make the interpolation molecules be
	 the ``same size and shape'' by padding them with
	 zero entries to bring them all up to the size of
	 the largest molecule used anywhere in the grid,
	 but this would be very inefficient if many molecules
	 were padded in this way.
	 }%%%
\item	If this interpolator supports computing derivatives
	as described in section~\ref{AEIThorns/AEILocalInterp/sect-derivatives},
	does the interpolation molecule size and/or shape depend
	on \verb|operation_codes[]|?
\item	Does the interpolation molecule size and/or shape depend
	on the actual floating-point values being interpolated?
	(Examples of this might include ENO (essentially nonoscillatory)
	and/or TVD (total-variation diminishing) interpolators for
	hydrodynamics calculations.)
\item	Do the actual floating-point values of the Jacobian depend
	on the actual floating-point values being interpolated?
	Equivalently, is the interpolation nonlinear?
\end{itemize}

Because the different cases differ so much in the amount of information
required to describe the Jacobian, it's hard to define a single API
which covers all cases without burdening the simpler cases with
excessive complexity.  Instead, the interpolator defines a
Jacobian-structure--query API to determine which case applies,
together with (conceptually) several different APIs for the
different cases.  (At the moment only a single case is implemented.)

%%%%%%%%%%%%%%%%%%%%

\subsubsection{Determining the Jacobian's structure}
\label{AEIThorns/AEILocalInterp/sect-Jacobian/structure}

The following parameter-table entries may be used to query which
of the different Jacobian-structure cases applies:

The parameter \verb|molecule_family| may may be used to query what
molecule family is being used.  This is described in detail in
section~\ref{AEIThorns/AEILocalInterp/sect-molecule-family}.

If the interpolation molecule size and/or shape vary with the
interpolation coordinates, the interpolator sets the parameter
\begin{verbatim}
CCTK_INT MSS_is_fn_of_interp_coords;
\end{verbatim}
to 1.  Otherwise (\ie{} if the interpolation molecule size and shape
are independent of the interpolation coordinates) it should set this
parameter to 0.

If the interpolator supports computing derivatives as described in
section~\ref{AEIThorns/AEILocalInterp/sect-derivatives}, {\em and\/}
if the interpolation molecule's size and/or shape varies with
\verb|operation_codes[]|, the interpolator sets the parameter
\begin{verbatim}
CCTK_INT MSS_is_fn_of_which_operation;
\end{verbatim}
to 1.  Otherwise (\ie{} if the interpolator doesn't support computing
derivatives, or if the interpolator does support computing derivatives
but the interpolation molecule size and shape are independent of the
\verb|operation_code[]| values), the interpolator sets this parameter
to 0.  Note that this query tests whether the molecule size and/or shape
depend on \verb|operation_codes[]| in general, independent of whether
there are in fact any distinct values (or even any values at all) passed
in \verb|operation_codes[]| in this particular interpolator call.  In
other words, this is a query about the basic design of the interpolator,
not about this particular call.

If the interpolation molecule's size and/or shape varies with the
actual floating-point values of the input arrays, the interpolator
sets the parameter
\begin{verbatim}
CCTK_INT MSS_is_fn_of_input_array_values;
\end{verbatim}
to 1.  Otherwise (\ie{} if the interpolation molecule size and shape
are independent of the input array values; this is a necessary, but
not sufficient, condition for the interpolation to be linear), the
interpolator sets this parameter to 0.

If the actual floating-point values of the
Jacobian~$(\ref{AEIThorns/AEILocalInterp/eqn-Jacobian})$
(for a given \verb|out|, \verb|in|, and \verb|pt|) depend on the actual
floating-point values of the input arrays (\ie{} if the interpolation
is nonlinear), the interpolator sets the parameter
\begin{verbatim}
CCTK_INT Jacobian_is_fn_of_input_array_values;
\end{verbatim}
to 1.  Otherwise (\ie{} if the interpolation is linear) the interpolator
sets this parameter to 0.

The current implementation always sets
\begin{verbatim}
MSS_is_fn_of_interp_coords = 0
MSS_is_fn_of_which_operation = 0
MSS_is_fn_of_input_array_values = 0
Jacobian_is_fn_of_input_array_values = 0
\end{verbatim}

%%%%%%%%%%%%%%%%%%%%

\subsubsection{Fixed-Size Hypercube-Shaped Molecules}
\label{AEIThorns/AEILocalInterp/sect-Jacobian/fixed-sized-hypercube}

The simplest case (and the only one for which we have defined an
API at present) is when the molecules are hypercube-shaped and of
(typically small) fixed size, independent of the interpolation
coordinates and the actual floating-point values in the input arrays
(though presumably depending on the interpolation order and on
\verb|operation_code|).  In other words, this case applies if
(and only if) the Jacobian structure information described in
section~\ref{AEIThorns/AEILocalInterp/sect-Jacobian/structure}
returns
\begin{verbatim}
MSS_is_fn_of_interp_coords = 0
MSS_is_fn_of_which_operation = 0
MSS_is_fn_of_input_array_values = 0
Jacobian_is_fn_of_input_array_values = 0
\end{verbatim}
(These are precisely the values set by the current implementation.)
In the rest of this section we describe the query API for this case.

The following parameters may be used to query the molecule size:
\begin{verbatim}
CCTK_INT const molecule_min_m[N_dims];
CCTK_INT const molecule_max_m[N_dims];
\end{verbatim}
The semantics of these are that if both of these keys are present
(the values and datatypes don't matter), then the interpolator will
(re)set the values to give the (inclusive) minimum and maximum
\verb|m|~molecule coordinates.  (Note that either both of these
keys should be present, or neither should be present.  This
simplifies the logic in the interpolator slightly.)

The following parameter may be used to query the molecule positions:
\begin{verbatim}
CCTK_INT* const molecule_positions[N_dims];
\end{verbatim}
The semantics of this is that the caller should set
\verb|molecule_positions[]| to an array of \verb|N_dims| pointers to
(caller-supplied) arrays of \verb|N_interp_points| \verb|CCTK_INT|s
each.  If this key exists, then the interpolator will store the
molecule positions in the pointed-to arrays.

The following parameters may be used to query the
Jacobian~$(\ref{AEIThorns/AEILocalInterp/eqn-Jacobian})$ itself:
\begin{verbatim}
CCTK_REAL* const Jacobian_pointer[N_output_arrays];
const CCTK_INT   Jacobian_offset [N_output_arrays];   # optional, default=all 0

/* the next 3 table entries are shared by all Jacobians */
const CCTK_INT Jacobian_interp_point_stride;
const CCTK_INT Jacobian_m_strides[N_dims];
const CCTK_INT Jacobian_part_stride;                  # optional, default=1
\end{verbatim}
If \verb|Jacobian_pointer| is present in the table, then
\verb|Jacobian_interp_point_stride| and \verb|Jacobian_m_strides|
must also be present.  For each \verb|out| where
\verb|Jacobian_pointer[out] != NULL|,%%%
\footnote{%%%
	 That is, setting {\tt Jacobian\_pointer[out] = NULL}
	 supresses the Jacobian query.  If (as is often the
	 case) the Jacobian is independent of {\tt out}, then
	 it's a good idea to avoid unnecessary computations by
	 supressing the queries in this manner for any remaining
	 output arrays.
	 }%%%
{} the interpolator would then store the
Jacobian~$(\ref{AEIThorns/AEILocalInterp/eqn-Jacobian})$ in
\begin{verbatim}
Jacobian_pointer[out][offset
                      + pt*Jacobian_interp_point_stride
                      + mi*stride_i + mj*stride_j + mk*stride_k
                      + part*Jacobian_part_stride]
\end{verbatim}
where
\begin{verbatim}
offset = Jacobian_offset[out]
(stride_i,stride_j,stride_k) = Jacobian_m_strides[]
\end{verbatim}
and where \verb|part| is 0 for real values and the real parts of complex
values, and 1 for the imaginary parts of complex values.

By appropriately setting the various stride parameters, this allows
a fairly wide variety of possible storage layouts for the Jacobian.

An example may help to clarify this:  Suppose we have a 1-D grid
with 11~grid points, with integer subscripts~0 through~10 inclusive,
and interpolation coordinates given by \verb|coord_origin = 0.0|
and \verb|coord_delta = 0.1|.
Suppose further that we're doing Lagrange polynomial interpolation,
with \verb|order = 2| and hence
(by table~\ref{AEIThorns/AEILocalInterp/tab-molecule-size+centering})
using 3-point molecules.
Finally, suppose that we query the Jacobian molecule positions for
the \verb|N_interp_points=14| interpolation points 0.0, 0.04, 0.06,
0.10, 0.14, 0.16, 0.20, 0.80, 0.84, 0.86, 0.90, 0.94, 0.96, 1.00.
Then the queries might return
\begin{verbatim}
interp_molecule_min_m = -1
interp_molecule_max_m = +1
                                         /* interp_x      molecule     */
interp_molecule_positions[0][ 0] =  1    /*   0.00     [0.0, 0.1, 0.2] */
interp_molecule_positions[0][ 1] =  1    /*   0.04     [0.0, 0.1, 0.2] */
interp_molecule_positions[0][ 2] =  1    /*   0.06     [0.0, 0.1, 0.2] */
interp_molecule_positions[0][ 3] =  1    /*   0.10     [0.0, 0.1, 0.2] */
interp_molecule_positions[0][ 4] =  1    /*   0.14     [0.0, 0.1, 0.2] */
interp_molecule_positions[0][ 5] =  2    /*   0.16     [0.1, 0.2, 0.3] */
interp_molecule_positions[0][ 6] =  2    /*   0.20     [0.1, 0.2, 0.3] */
interp_molecule_positions[0][ 7] =  8    /*   0.80     [0.7, 0.8, 0.9] */
interp_molecule_positions[0][ 8] =  8    /*   0.84     [0.7, 0.8, 0.9] */
interp_molecule_positions[0][ 9] =  9    /*   0.86     [0.8, 0.9, 1.0] */
interp_molecule_positions[0][10] =  9    /*   0.90     [0.8, 0.9, 1.0] */
interp_molecule_positions[0][11] =  9    /*   0.94     [0.8, 0.9, 1.0] */
interp_molecule_positions[0][12] =  9    /*   0.96     [0.8, 0.9, 1.0] */
interp_molecule_positions[0][13] =  9    /*   1.00     [0.8, 0.9, 1.0] */
\end{verbatim}

%%%%%%%%%%%%%%%%%%%%%%%%%%%%%%%%%%%%%%%%

\subsection{Smoothing}

The way the generalized polynomial interpolator is implemented it's
easy to also do Savitzky-Golay smoothing.%%%
\footnote{%%%
	 See section~14.8 of Numerical Recipes
	 (2nd edition) for a general discussion of
	 Savitzky-Golay smoothing.
	 }%%%
{}  This is best described by way of an example:  Suppose we're doing
1-D cubic interpolation, using
(by table~\ref{AEIThorns/AEILocalInterp/tab-molecule-size+centering})
4-point molecules.  In other words, at each interpolation point we
use a cubic interpolation polynomial fitted to 4~surrounding data points.
For Savitzky-Golay smoothing, we would instead {\em least-squares fit\/}
a cubic polynomial to some {\em larger\/} number of surrounding data
points.  This combines interpolation with smoothing, so there's less
amplification of noise in the input data in the interpolation outputs.

The optional input
\begin{verbatim}
const CCTK_INT smoothing;
\end{verbatim}
specifies how much (how many points) to enlarge the interpolation
molecule for this.  The default is 0 (no smoothing).  1 would mean to
enlarge the molecule by 1 point (\eg{} to use a 5-point molecule instead
of the usual 4-point one for cubic interpolation).  2 would mean to
enlarge by 2 points, (\eg{} to use a 6-point molecule for cubic
interpolation).  Etc etc.

Note that in $>1$~dimension, the maximum-degree Lagrange interpolation
already uses more data points than the number of free parameters in
the interpolating polynomials, \ie{} it already does some Savitzky-Golay
smoothing.  For example, in 2-D a generic cubic polynomial
$f(x,y) = \sum_{i+j \le 3} c_{ij} x^i y^j$ has 10 free parameters
$c_{ij}$, which we least-squares fit to the 16 data points in the
$4 \times 4$ molecule.

Savitzky-Golay smoothing is basically free apart from the increase in
the molecule size, e.g. a \verb|smoothing|=2 cubic interpolation has
exactly the same cost as any other 6-point--molecule interpolation.

The current implementation has all the framework for smoothing, but
only the \verb|smoothing=0| case is implemented at the moment.

%%%%%%%%%%%%%%%%%%%%%%%%%%%%%%%%%%%%%%%%

\subsection{Debugging}

Setting the optional parameter
\begin{verbatim}
const CCTK_INT debug;
\end{verbatim}
to a positive value turns on various debug printing inside the
interpolator.  This is intended for debugging the interpolator itself;
you need to look at the interpolator source code to interpret the output.

%%%%%%%%%%%%%%%%%%%%%%%%%%%%%%%%%%%%%%%%

\subsection{Logging}

This thorn has a Boolean parameter \verb|log_interp_coords|.
If this is set to \verb|true|, this thorn will write a detailed log
file giving the grid information and interpolation coordinates for
each interpolator call.  Each processor will write a separate log file,
with the file name determined via
\begin{verbatim}
snprintf(buffer, length,
         "AEILocalInterp.proc%d.log"
         CCTK_MyProc(NULL))
\end{verbatim}

Each file begins with a header comment describing the file format in
detail.  After the header comment, there is one line in the file for
each call on \verb|CCTK_InterpLocalUniform()| which is handled by this thorn
(with \verb|log_interp_coords| set to \verb|true|).

Note that these log files are typically quite large, and writing them         
may significantly slow the simulation -- you should probably only use
this option for debugging purposes.

%%%%%%%%%%%%%%%%%%%%%%%%%%%%%%%%%%%%%%%%%%%%%%%%%%%%%%%%%%%%%%%%%%%%%%%%%%%%%%%%

\section{Supressing Interpolation}

Sometimes you may want to call the interpolator, but not actually
do any interpolation for some or all of the arrays:
\begin{itemize}
\item	you may be doing a query
\item	you may be just checking values in the parameter table
\item	you may want to interpolate varying subsets of some set
	of arrays
\item	you may have no interpolation points, yet still want to
	do the interpolator call for other reasons
\end{itemize}

In such cases there are several possible ways you can supress some
or all of the interpolations:
\begin{description}
\item[\mathversion{bold}
      Set $\text{\tt N\_interp\_points} = 0$:]\mbox{}\\
	This supresses all interpolation.  In this case
	\verb|interp_coords| may also be \verb|NULL| if desired.
\item[\mathversion{bold}
      Set $\text{\tt N\_input\_arrays} = 0$ and
      $\text{\tt N\_output\_arrays} = 0$:]\mbox{}\\
	This suppresses all interpolation.  However, note that
	some parameter-table entries like \verb|operand_indices|
	and \verb|operation_codes| are specified as being arrays
	whose size depends on \verb|N_output_arrays|, and it's an
	error for these table entries to be present with the wrong
	size (in this case, with any nonzero size).
\item[\mathversion{bold}
      Set $\text{\tt output\_arrays[out]} = \text{\tt NULL}$:]\mbox{}\\
	This supresses the interpolation for this particular
	output array.  This is probably the cleanest way of
	selectively turning individual arrays on and off.
	If $\verb|output_arrays[out]| = \verb|NULL|$ and
	the input arrays aren't needed for any queries you're doing,
	then it's useful to also set the corresponding
	$\verb|input_arrays[in]| = \verb|NULL|$, to supress the
	interpolator fetching data (which in this case will never be used)
	from the input arrays.
\end{description}

Note that the combination
$\verb|input_arrays[in]| = \verb|NULL|$ and
$\verb|output_arrays[out]| \ne \verb|NULL|$ is an error: conceptually
you're asking for an interpolation (the output array pointer is
non-\verb|NULL|) but not supplying any input data (the input array
pointer is \verb|NULL|).

%%%%%%%%%%%%%%%%%%%%%%%%%%%%%%%%%%%%%%%%%%%%%%%%%%%%%%%%%%%%%%%%%%%%%%%%%%%%%%%%

\section{Implementation}

This interpolator's basic design is to use separate specialized code
for each combination of
\begin{verbatim}
   (N_dims, molecule_family, order, smoothing)
\end{verbatim}
\ie{} in practice for each distinct choice of interpolation molecule.
Maple is used to generate all the interpolation coefficients.
The C preprocessor is then used to generate all the specialized code
from a single master ``template'' file.  The template code uses
\verb|#ifdef|s to handle lower dimensions with no extra overhead,
\eg{} 1-D/2-D interpolation is basically just as efficient as in
a purpose-built 1-D/2-D interpolator, with no extra overhead imposed
by the interpolator also supporting higher-dimensional interpolation.

The Maple code which generates the interpolation coefficients is quite
general-purpose, and can handle an arbitrary dimensionality and molecule
size/shape.  Generating new coefficients can be rather time-consuming,
though, \eg{} the current coefficients for 3-D for orders~1-4
take about 8~cpu minutes to generate using Maple~7 on a 1.7~GHz~P4.

Note that when compiling the code in the directory
\verb|src/GeneralizedPolynomial-Uniform| of this thorn, you may get
compiler warnings about casts discarding \verb|const| qualifiers from
pointers in (the \verb|#include|-ed file) \verb|template.c|.  Don't
worry -- the code is actually ok.%%%
\footnote{%%%
	 The warnings are really a design bug in C's
	 const-qualifier rules when pointers point to
	 other pointers.  See question~11.10 in the
	 (excellent!) {\tt comp.lang.c} online FAQ
	 ({\tt http://www.eskimo.com/~scs/C-faq/top.html})
	 for further discussion.  gcc~2.95.3 gives the
	 spurious warnings, but gcc~3.2.1 doesn't, nor
	 does the Intel C compiler in any version I've
	 tried.  Also, \Cplusplus{} has fixed the problems
	 in C's const-qualifier rules, so the same code
	 compiled as \Cplusplus{} shouldn't give any
	 warnings (and doesn't on the compilers I've tried).
	 }%%%
{}  You may also get compiler warnings about unused variables in
this same file; again don't worry, the code is ok.

See the \verb|README| file in the source code directory
\verb|AEILocalInterp/src/| and in its subdirectories for further details
on the implementation.

%%%%%%%%%%%%%%%%%%%%%%%%%%%%%%%%%%%%%%%%%%%%%%%%%%%%%%%%%%%%%%%%%%%%%%%%%%%%%%%%

\section{Examples}

%%%%%%%%%%%%%%%%%%%%%%%%%%%%%%%%%%%%%%%%

\subsection{A Simple Example}

Here's a simple example in C, of interpolating a \verb|CCTK_REAL| and a
\verb|CCTK_COMPLEX| $10 \times 20$ 2-D array, at 5 interpolation points,
using cubic interpolation.

{\bf Note:}
Since C allows arrays to be initialized only if the initializer
values are compile-time constants, we have to declare the
\verb|interp_coords[]|, \verb|input_arrays[]|, and \verb|output_arrays[]|
arrays as non-\verb|const|, and set their values with ordinary
(run-time) assignment statements.  In \Cplusplus, there's no
restriction on initializer values, so we could declare the arrays
\verb|const| and initialize them as part of their declarations.

\begin{verbatim}
#define N_DIMS   2
#define N_INTERP_POINTS   5
#define N_INPUT_ARRAYS    2
#define N_OUTPUT_ARRAYS   2

/* (x,y) coordinates of data grid points */
#define X_ORIGIN   ...
#define X_DELTA    ...
#define Y_ORIGIN   ...
#define Y_DELTA    ...
const CCTK_REAL origin[N_DIMS] = { X_ORIGIN, Y_ORIGIN };
const CCTK_REAL delta [N_DIMS] = { X_DELTA,  Y_DELTA  };

/* (x,y) coordinates of interpolation points */
const CCTK_REAL interp_x[N_INTERP_POINTS];
const CCTK_REAL interp_y[N_INTERP_POINTS];
const void* interp_coords[N_DIMS];              /* see note above */

/* input arrays */
/* ... note Cactus uses Fortran storage ordering, i.e. X is contiguous */
#define NX   10
#define NY   20
const CCTK_REAL    input_real   [NY][NX];
const CCTK_COMPLEX input_complex[NY][NX];
const CCTK_INT input_array_dims[N_DIMS] = { NX, NY };
const CCTK_INT input_array_type_codes[N_INPUT_ARRAYS]
        = { CCTK_VARIABLE_REAL, CCTK_VARIABLE_COMPLEX };
const void* input_arrays[N_INPUT_ARRAYS];       /* see note above */

/* output arrays */
CCTK_REAL    output_real   [N_INTERP_POINTS];
CCTK_COMPLEX output_complex[N_INTERP_POINTS];
const CCTK_INT output_array_type_codes[N_OUTPUT_ARRAYS]
        = { CCTK_VARIABLE_REAL, CCTK_VARIABLE_COMPLEX };
void* output_arrays[N_OUTPUT_ARRAYS];           /* see note above */

int operator_handle, param_table_handle;
operator_handle = CCTK_InterpHandle("my interpolation operator");
if (operator_handle < 0)
        CCTK_WARN(-1, "can't get interpolation handle!");
param_table_handle = Util_TableCreateFromString("order=3");
if (param_table_handle < 0)
        CCTK_WARN(-1, "can't create parameter table!");

/* initialize the rest of the parameter arrays */
interp_coords[0] = (const void *) interp_x;
interp_coords[1] = (const void *) interp_y;
input_arrays [0] = (const void *) input_real;
input_arrays [1] = (const void *) input_complex;
output_arrays[0] = (      void *) output_real;
output_arrays[1] = (      void *) output_complex;

/* do the actual interpolation, and check for error returns */
if (CCTK_InterpLocalUniform(N_DIMS,
                            operator_handle, param_table_handle,
                            origin, delta,
                            N_INTERP_POINTS,
                               CCTK_VARIABLE_REAL,
                               interp_coords,
                            N_INPUT_ARRAYS,
                               input_array_dims,
                               input_array_type_codes,
                               input_arrays,
                            N_OUTPUT_ARRAYS,
                               output_array_type_codes,
                               output_arrays) < 0)
        CCTK_WARN(-1, "error return from interpolator!");
\end{verbatim}

%%%%%%%%%%%%%%%%%%%%%%%%%%%%%%%%%%%%%%%%

\subsection{An Example of Interpolating Derivatives}
\label{AEIThorns/AEILocalInterp/sect-example-derivatives}

In this example we interpolate the 3-metric and its 1st~derivatives
at a set of interpolation points on a 2-sphere.

{\bf Note:}
Since we're not using \Cplusplus, we again declare the pointer arrays
non-\verb|const|, and ``initialize'' them with ordinary (run-time)
assignment statements.  However, in this example, for greater clarity
we place these assignment statements right after the declarations.
Since C allows declarations only at the start of a \verb|{ }| block,
not in the middle of a block, we nest the rest of the program in extra
blocks (with the \verb|{ }| indented 2 spaces to distinguish them from
``normal'' \verb|{ }| pairs) to allow for further declarations.  The
reader will have to decide for herself whether this style is more or
less ugly than separating the declarations and initializations of the
pointer arrays.

\begin{verbatim}
#define N_DIMS        3

/* interpolation points */
#define N_INTERP_POINTS 1000
const CCTK_REAL interp_x[N_INTERP_POINTS],
                interp_y[N_INTERP_POINTS],
                interp_z[N_INTERP_POINTS];
const void* interp_coords[N_DIMS];              /* see note above */
interp_coords[0] = (const void *) interp_x;
interp_coords[1] = (const void *) interp_y;
interp_coords[2] = (const void *) interp_z;

/* dimensions of the data grid */
#define NX   30
#define NY   40
#define NZ   50

  {
/* input arrays */
/* n.b. we use Fortran storage order: X is contiguous, Z least so */
#define N_INPUT_ARRAYS  6
const CCTK_REAL gxx_3D[NZ][NY][NX], gxy_3D[NZ][NY][NX], gxz_3D[NZ][NY][NX],
                                    gyy_3D[NZ][NY][NX], gyz_3D[NZ][NY][NX],
                                                        gzz_3D[NZ][NY][NX];

const CCTK_INT input_array_dims[N_DIMS] = {NX, NY, NZ};
const CCTK_INT input_array_type_codes[N_INPUT_ARRAYS]
        = { CCTK_VARIABLE_REAL, CCTK_VARIABLE_REAL, CCTK_VARIABLE_REAL,
                                CCTK_VARIABLE_REAL, CCTK_VARIABLE_REAL,
                                                    CCTK_VARIABLE_REAL };
const void* input_arrays[N_INPUT_ARRAYS];       /* see note above */
input_arrays[0] = (const void *) gxx_3D;
input_arrays[1] = (const void *) gxy_3D;
input_arrays[2] = (const void *) gxz_3D;
input_arrays[3] = (const void *) gyy_3D;
input_arrays[4] = (const void *) gyz_3D;
input_arrays[5] = (const void *) gzz_3D;

  {
/* output arrays */
/* (for best efficiency we group all operations on a given input together) */
#define N_OUTPUT_ARRAYS 24
CCTK_REAL
   gxx[N_INTERP_POINTS],
      dx_gxx[N_INTERP_POINTS], dy_gxx[N_INTERP_POINTS], dz_gxx[N_INTERP_POINTS],
   gxy[N_INTERP_POINTS],
      dx_gxy[N_INTERP_POINTS], dy_gxy[N_INTERP_POINTS], dz_gxy[N_INTERP_POINTS],
   gxz[N_INTERP_POINTS],
      dx_gxz[N_INTERP_POINTS], dy_gxz[N_INTERP_POINTS], dz_gxz[N_INTERP_POINTS],
   gyy[N_INTERP_POINTS],
      dx_gyy[N_INTERP_POINTS], dy_gyy[N_INTERP_POINTS], dz_gyy[N_INTERP_POINTS],
   gyz[N_INTERP_POINTS],
      dx_gyz[N_INTERP_POINTS], dy_gyz[N_INTERP_POINTS], dz_gyz[N_INTERP_POINTS],
   gzz[N_INTERP_POINTS],
      dx_gzz[N_INTERP_POINTS], dy_gzz[N_INTERP_POINTS], dz_gzz[N_INTERP_POINTS];
const CCTK_INT output_array_type_codes[N_OUTPUT_ARRAYS]
   = { CCTK_VARIABLE_REAL,
          CCTK_VARIABLE_REAL, CCTK_VARIABLE_REAL, CCTK_VARIABLE_REAL,
       CCTK_VARIABLE_REAL,
          CCTK_VARIABLE_REAL, CCTK_VARIABLE_REAL, CCTK_VARIABLE_REAL,
       CCTK_VARIABLE_REAL,
          CCTK_VARIABLE_REAL, CCTK_VARIABLE_REAL, CCTK_VARIABLE_REAL,
       CCTK_VARIABLE_REAL,
          CCTK_VARIABLE_REAL, CCTK_VARIABLE_REAL, CCTK_VARIABLE_REAL,
       CCTK_VARIABLE_REAL,
          CCTK_VARIABLE_REAL, CCTK_VARIABLE_REAL, CCTK_VARIABLE_REAL,
       CCTK_VARIABLE_REAL,
          CCTK_VARIABLE_REAL, CCTK_VARIABLE_REAL, CCTK_VARIABLE_REAL };
void* output_arrays[N_OUTPUT_ARRAYS];           /* see note above */
output_arrays[ 0] = (void *)    gxx;
output_arrays[ 1] = (void *) dx_gxx;
output_arrays[ 2] = (void *) dy_gxx;
output_arrays[ 3] = (void *) dz_gxx;
output_arrays[ 4] = (void *)    gxy;
output_arrays[ 5] = (void *) dx_gxy;
output_arrays[ 6] = (void *) dy_gxy;
output_arrays[ 7] = (void *) dz_gxy;
output_arrays[ 8] = (void *)    gxz;
output_arrays[ 9] = (void *) dx_gxz;
output_arrays[10] = (void *) dy_gxz;
output_arrays[11] = (void *) dz_gxz;
output_arrays[12] = (void *)    gyy;
output_arrays[13] = (void *) dx_gyy;
output_arrays[14] = (void *) dy_gyy;
output_arrays[15] = (void *) dz_gyy;
output_arrays[16] = (void *)    gyz;
output_arrays[17] = (void *) dx_gyz;
output_arrays[18] = (void *) dy_gyz;
output_arrays[19] = (void *) dz_gyz;
output_arrays[20] = (void *)    gzz;
output_arrays[21] = (void *) dx_gzz;
output_arrays[22] = (void *) dy_gzz;
output_arrays[23] = (void *) dz_gzz;

  {
/* integer codes to specify the derivatives */
const CCTK_INT operand_indices[N_OUTPUT_ARRAYS]
   = { 0, 0, 0, 0,
       1, 1, 1, 1,
       2, 2, 2, 2,
       3, 3, 3, 3,
       4, 4, 4, 4,
       5, 5, 5, 5 };
#define DERIV(x)   x
const CCTK_INT operation_codes[N_OUTPUT_ARRAYS]
   = { DERIV(0), DERIV(1), DERIV(2), DERIV(3),
       DERIV(0), DERIV(1), DERIV(2), DERIV(3),
       DERIV(0), DERIV(1), DERIV(2), DERIV(3),
       DERIV(0), DERIV(1), DERIV(2), DERIV(3),
       DERIV(0), DERIV(1), DERIV(2), DERIV(3),
       DERIV(0), DERIV(1), DERIV(2), DERIV(3) };

int operator_handle, param_table_handle;
operator_handle = CCTK_InterpHandle("my interpolation operator");
if (operator_handle < 0)
        CCTK_WARN(-1, "can't get interpolation handle!");

param_table_handle = Util_TableCreate(UTIL_TABLE_FLAGS_DEFAULT);
if (param_table_handle < 0)
        CCTK_WARN(-1, "can't create parameter table!");
if (Util_TableSetInt(param_table_handle, 3, "order") < 0)
        CCTK_WARN(-1, "can't set order in parameter table!");
if (Util_TableSetIntArray(param_table_handle,
                          N_OUTPUT_ARRAYS, operand_indices,
                          "operand_indices") < 0)
        CCTK_WARN(-1, "can't set operand_indices array in parameter table!");
if (Util_TableSetIntArray(param_table_handle,
                          N_OUTPUT_ARRAYS, operation_codes,
                          "operation_codes") < 0)
        CCTK_WARN(-1, "can't set operation_codes array in parameter table!");

if (CCTK_InterpLocalUniform(N_DIMS,
                            operator_handle, param_table_handle,
                            origin, delta,
                            N_INTERP_POINTS,
                               CCTK_VARIABLE_REAL,
                               interp_coords,
                            N_INPUT_ARRAYS,
                               input_array_dims,
                               input_array_type_codes,
                               input_arrays,
                            N_OUTPUT_ARRAYS,
                               output_array_type_codes,
                               output_arrays) < 0)
        CCTK_WARN(-1, "error return from interpolator!");
  }
  }
  }
\end{verbatim}

%%%%%%%%%%%%%%%%%%%%%%%%%%%%%%%%%%%%%%%%%%%%%%%%%%%%%%%%%%%%%%%%%%%%%%%%%%%%%%%%

\section{Acknowledgments}

Thanks to Thomas Radke for our ongoing collaboration on Cactus
interpolation,
to Ian Hawke and Erik Schnetter for helpful comments on the documentation,
to Tom Goodale and Thomas Radke for many useful design discussions,
to Erik Schnetter for bug reports,
and to all the Cactus crew for a great infrastructure!

%%%%%%%%%%%%%%%%%%%%%%%%%%%%%%%%%%%%%%%%%%%%%%%%%%%%%%%%%%%%%%%%%%%%%%%%%%%%%%%%

% Do not delete next line
% END CACTUS THORNGUIDE



\section{Parameters} 


\parskip = 0pt

\setlength{\tableWidth}{160mm}

\setlength{\paraWidth}{\tableWidth}
\setlength{\descWidth}{\tableWidth}
\settowidth{\maxVarWidth}{log\_interp\_coords}

\addtolength{\paraWidth}{-\maxVarWidth}
\addtolength{\paraWidth}{-\columnsep}
\addtolength{\paraWidth}{-\columnsep}
\addtolength{\paraWidth}{-\columnsep}

\addtolength{\descWidth}{-\columnsep}
\addtolength{\descWidth}{-\columnsep}
\addtolength{\descWidth}{-\columnsep}
\noindent \begin{tabular*}{\tableWidth}{|c|l@{\extracolsep{\fill}}r|}
\hline
\multicolumn{1}{|p{\maxVarWidth}}{log\_interp\_coords} & {\bf Scope:} private & BOOLEAN \\\hline
\multicolumn{3}{|p{\descWidth}|}{{\bf Description:}   {\em should we log the grid min(delta)max and the interpolation		   coordinates for each call on the interpolator?}} \\
\hline & & {\bf Default:} false \\\hline
\end{tabular*}

\vspace{0.5cm}\parskip = 10pt 

\section{Interfaces} 


\parskip = 0pt

\vspace{3mm} \subsection*{General}

\noindent {\bf Implements}: 

aeilocalinterp
\vspace{2mm}

\vspace{5mm}\parskip = 10pt 

\section{Schedule} 


\parskip = 0pt


\noindent This section lists all the variables which are assigned storage by thorn Numerical/AEILocalInterp.  Storage can either last for the duration of the run ({\bf Always} means that if this thorn is activated storage will be assigned, {\bf Conditional} means that if this thorn is activated storage will be assigned for the duration of the run if some condition is met), or can be turned on for the duration of a schedule function.


\subsection*{Storage}NONE
\subsection*{Scheduled Functions}
\vspace{5mm}

\noindent {\bf CCTK\_STARTUP} 

\hspace{5mm} aeilocalinterp\_u\_startup 

\hspace{5mm}{\it register cctk\_interplocaluniform() interpolation operators } 


\hspace{5mm}

 \begin{tabular*}{160mm}{cll} 
~ & After:  & driver\_startup \\ 
~ & Language:  & c \\ 
~ & Type:  & function \\ 
\end{tabular*} 



\vspace{5mm}\parskip = 10pt 
\end{document}
