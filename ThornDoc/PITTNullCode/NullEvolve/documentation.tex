% *======================================================================*
%  Cactus Thorn template for ThornGuide documentation
%  Author: Ian Kelley
%  Date: Sun Jun 02, 2002
%  $Header$                                                             
%
%  Thorn documentation in the latex file doc/documentation.tex 
%  will be included in ThornGuides built with the Cactus make system.
%  The scripts employed by the make system automatically include 
%  pages about variables, parameters and scheduling parsed from the 
%  relevent thorn CCL files.
%  
%  This template contains guidelines which help to assure that your     
%  documentation will be correctly added to ThornGuides. More 
%  information is available in the Cactus UsersGuide.
%                                                    
%  Guidelines:
%   - Do not change anything before the line
%       % START CACTUS THORNGUIDE",
%     except for filling in the title, author, date etc. fields.
%        - Each of these fields should only be on ONE line.
%        - Author names should be sparated with a \\ or a comma
%   - You can define your own macros are OK, but they must appear after
%     the START CACTUS THORNGUIDE line, and do not redefine standard 
%     latex commands.
%   - To avoid name clashes with other thorns, 'labels', 'citations', 
%     'references', and 'image' names should conform to the following 
%     convention:          
%       ARRANGEMENT_THORN_LABEL
%     For example, an image wave.eps in the arrangement CactusWave and 
%     thorn WaveToyC should be renamed to CactusWave_WaveToyC_wave.eps
%   - Graphics should only be included using the graphix package. 
%     More specifically, with the "includegraphics" command. Do
%     not specify any graphic file extensions in your .tex file. This 
%     will allow us (later) to create a PDF version of the ThornGuide
%     via pdflatex. |
%   - References should be included with the latex "bibitem" command. 
%   - Use \begin{abstract}...\end{abstract} instead of \abstract{...}
%   - Do not use \appendix, instead include any appendices you need as 
%     standard sections. 
%   - For the benefit of our Perl scripts, and for future extensions, 
%     please use simple latex.     
%
% *======================================================================* 
% 
% Example of including a graphic image:
%    \begin{figure}[ht]
% 	\begin{center}
%    	   \includegraphics[width=6cm]{/home/runner/work/tests/tests/arrangements/PITTNullCode/NullEvolve/doc/MyArrangement_MyThorn_MyFigure}
% 	\end{center}
% 	\caption{Illustration of this and that}
% 	\label{MyArrangement_MyThorn_MyLabel}
%    \end{figure}
%
% Example of using a label:
%   \label{MyArrangement_MyThorn_MyLabel}
%
% Example of a citation:
%    \cite{MyArrangement_MyThorn_Author99}
%
% Example of including a reference
%   \bibitem{MyArrangement_MyThorn_Author99}
%   {J. Author, {\em The Title of the Book, Journal, or periodical}, 1 (1999), 
%   1--16. {\tt http://www.nowhere.com/}}
%
% *======================================================================* 

% If you are using CVS use this line to give version information
% $Header$

\documentclass{article}

% Use the Cactus ThornGuide style file
% (Automatically used from Cactus distribution, if you have a 
%  thorn without the Cactus Flesh download this from the Cactus
%  homepage at www.cactuscode.org)
\usepackage{../../../../../doc/latex/cactus}

\newlength{\tableWidth} \newlength{\maxVarWidth} \newlength{\paraWidth} \newlength{\descWidth} \begin{document}

% The author of the documentation
\author{Y.Z. \textless yosef@raven.phyast.pitt.edu\textgreater}

% The title of the document (not necessarily the name of the Thorn)
\title{NullEvolve}

% the date your document was last changed, if your document is in CVS, 
% please use:
%    \date{$ $Date: 2004-07-30 16:43:52 -0400 (Fri, 30 Jul 2004) $ $}
\date{January 09 2003}

\maketitle

% Do not delete next line
% START CACTUS THORNGUIDE

% Add all definitions used in this documentation here 
%   \def\mydef etc

% Add an abstract for this thorn's documentation
\begin{abstract}

\end{abstract}

% The following sections are suggestive only.
% Remove them or add your own.

\section{Introduction}

\section{Physical System}

\section{Numerical Implementation}

\section{Using This Thorn}

\subsection{Obtaining This Thorn}

\subsection{Basic Usage}

\subsection{Special Behaviour}

\subsection{Interaction With Other Thorns}

\subsection{Examples}

\subsection{Support and Feedback}

\section{History}

\subsection{Thorn Source Code}

\subsection{Thorn Documentation}

\subsection{Acknowledgements}


\begin{thebibliography}{9}

\end{thebibliography}

% Do not delete next line
% END CACTUS THORNGUIDE



\section{Parameters} 


\parskip = 0pt

\setlength{\tableWidth}{160mm}

\setlength{\paraWidth}{\tableWidth}
\setlength{\descWidth}{\tableWidth}
\settowidth{\maxVarWidth}{n\_dissip\_zero\_outside\_eq}

\addtolength{\paraWidth}{-\maxVarWidth}
\addtolength{\paraWidth}{-\columnsep}
\addtolength{\paraWidth}{-\columnsep}
\addtolength{\paraWidth}{-\columnsep}

\addtolength{\descWidth}{-\columnsep}
\addtolength{\descWidth}{-\columnsep}
\addtolength{\descWidth}{-\columnsep}
\noindent \begin{tabular*}{\tableWidth}{|c|l@{\extracolsep{\fill}}r|}
\hline
\multicolumn{1}{|p{\maxVarWidth}}{dissip\_fudge} & {\bf Scope:} private & INT \\\hline
\multicolumn{3}{|p{\descWidth}|}{{\bf Description:}   {\em do we want to fudge around the inner boundary}} \\
\hline{\bf Range} & &  {\bf Default:} (none) \\\multicolumn{1}{|p{\maxVarWidth}|}{\centering 0:1} & \multicolumn{2}{p{\paraWidth}|}{0 no fudging, 1 fudging} \\\hline
\end{tabular*}

\vspace{0.5cm}\noindent \begin{tabular*}{\tableWidth}{|c|l@{\extracolsep{\fill}}r|}
\hline
\multicolumn{1}{|p{\maxVarWidth}}{dissip\_fudge\_maxx} & {\bf Scope:} private & REAL \\\hline
\multicolumn{3}{|p{\descWidth}|}{{\bf Description:}   {\em where in x to stop fudging}} \\
\hline{\bf Range} & &  {\bf Default:} .53 \\\multicolumn{1}{|p{\maxVarWidth}|}{\centering .5:.56} & \multicolumn{2}{p{\paraWidth}|}{ } \\\hline
\end{tabular*}

\vspace{0.5cm}\noindent \begin{tabular*}{\tableWidth}{|c|l@{\extracolsep{\fill}}r|}
\hline
\multicolumn{1}{|p{\maxVarWidth}}{dissip\_j} & {\bf Scope:} private & REAL \\\hline
\multicolumn{3}{|p{\descWidth}|}{{\bf Description:}   {\em dissipation factor for the J equation (time-integration)}} \\
\hline{\bf Range} & &  {\bf Default:} 0.0 \\\multicolumn{1}{|p{\maxVarWidth}|}{\centering 0.0:*} & \multicolumn{2}{p{\paraWidth}|}{positive} \\\hline
\end{tabular*}

\vspace{0.5cm}\noindent \begin{tabular*}{\tableWidth}{|c|l@{\extracolsep{\fill}}r|}
\hline
\multicolumn{1}{|p{\maxVarWidth}}{dissip\_jx} & {\bf Scope:} private & REAL \\\hline
\multicolumn{3}{|p{\descWidth}|}{{\bf Description:}   {\em dissipation factor for the J equation (radial marching)}} \\
\hline{\bf Range} & &  {\bf Default:} 0.0 \\\multicolumn{1}{|p{\maxVarWidth}|}{\centering 0.0:*} & \multicolumn{2}{p{\paraWidth}|}{positive} \\\hline
\end{tabular*}

\vspace{0.5cm}\noindent \begin{tabular*}{\tableWidth}{|c|l@{\extracolsep{\fill}}r|}
\hline
\multicolumn{1}{|p{\maxVarWidth}}{dissip\_mask\_type} & {\bf Scope:} private & KEYWORD \\\hline
\multicolumn{3}{|p{\descWidth}|}{{\bf Description:}   {\em type of dissipation mask}} \\
\hline{\bf Range} & &  {\bf Default:} one \\\multicolumn{1}{|p{\maxVarWidth}|}{\centering one} & \multicolumn{2}{p{\paraWidth}|}{set dissipation mask to 'one' everywhere} \\\multicolumn{1}{|p{\maxVarWidth}|}{see [1] below} & \multicolumn{2}{p{\paraWidth}|}{"set 2B=1-(p\^2+q\^2)+|1-(p 
\^2+q\^2)|"} \\\multicolumn{1}{|p{\maxVarWidth}|}{see [1] below} & \multicolumn{2}{p{\paraWidth}|}{"set 2B=rD0-(p\^2+q\^2)+|rD 
0-(p\^2+q\^2)|"} \\\multicolumn{1}{|p{\maxVarWidth}|}{see [1] below} & \multicolumn{2}{p{\paraWidth}|}{"set 2B=min(1, (rD0-(p\^2+q\^2)+|rD0- 
(p\^2+q\^2)|) / (rD0-1) )"} \\\multicolumn{1}{|p{\maxVarWidth}|}{see [1] below} & \multicolumn{2}{p{\paraWidth}|}{"set 2B=min(1, (rD0-(p\^2+q\^2)+|rD0- 
(p\^2+q\^2)|) / (rD0-rD1) )"} \\\hline
\end{tabular*}

\vspace{0.5cm}\noindent {\bf [1]} \noindent \begin{verbatim}zero at eq, one at pole\end{verbatim}\noindent {\bf [1]} \noindent \begin{verbatim}zero at rD0, one at pole\end{verbatim}\noindent {\bf [1]} \noindent \begin{verbatim}zero at rD0, one at eq\end{verbatim}\noindent {\bf [1]} \noindent \begin{verbatim}zero at rD0, one at rD1\end{verbatim}\noindent \begin{tabular*}{\tableWidth}{|c|l@{\extracolsep{\fill}}r|}
\hline
\multicolumn{1}{|p{\maxVarWidth}}{dissip\_q} & {\bf Scope:} private & REAL \\\hline
\multicolumn{3}{|p{\descWidth}|}{{\bf Description:}   {\em dissipation factor for the Q equation (the auxiliary of U)}} \\
\hline{\bf Range} & &  {\bf Default:} 0.0 \\\multicolumn{1}{|p{\maxVarWidth}|}{\centering 0.0:*} & \multicolumn{2}{p{\paraWidth}|}{positive} \\\hline
\end{tabular*}

\vspace{0.5cm}\noindent \begin{tabular*}{\tableWidth}{|c|l@{\extracolsep{\fill}}r|}
\hline
\multicolumn{1}{|p{\maxVarWidth}}{dissip\_w} & {\bf Scope:} private & REAL \\\hline
\multicolumn{3}{|p{\descWidth}|}{{\bf Description:}   {\em dissipation factor for the W equation}} \\
\hline{\bf Range} & &  {\bf Default:} 0.0 \\\multicolumn{1}{|p{\maxVarWidth}|}{\centering 0.0:*} & \multicolumn{2}{p{\paraWidth}|}{positive} \\\hline
\end{tabular*}

\vspace{0.5cm}\noindent \begin{tabular*}{\tableWidth}{|c|l@{\extracolsep{\fill}}r|}
\hline
\multicolumn{1}{|p{\maxVarWidth}}{id\_amp} & {\bf Scope:} private & REAL \\\hline
\multicolumn{3}{|p{\descWidth}|}{{\bf Description:}   {\em Amplitude of compact J pulse}} \\
\hline{\bf Range} & &  {\bf Default:} .00001 \\\multicolumn{1}{|p{\maxVarWidth}|}{\centering 0.:*} & \multicolumn{2}{p{\paraWidth}|}{An y positive number} \\\hline
\end{tabular*}

\vspace{0.5cm}\noindent \begin{tabular*}{\tableWidth}{|c|l@{\extracolsep{\fill}}r|}
\hline
\multicolumn{1}{|p{\maxVarWidth}}{id\_l} & {\bf Scope:} private & INT \\\hline
\multicolumn{3}{|p{\descWidth}|}{{\bf Description:}   {\em l mode}} \\
\hline{\bf Range} & &  {\bf Default:} 2 \\\multicolumn{1}{|p{\maxVarWidth}|}{\centering 2:*} & \multicolumn{2}{p{\paraWidth}|}{2 or larger} \\\hline
\end{tabular*}

\vspace{0.5cm}\noindent \begin{tabular*}{\tableWidth}{|c|l@{\extracolsep{\fill}}r|}
\hline
\multicolumn{1}{|p{\maxVarWidth}}{id\_m} & {\bf Scope:} private & INT \\\hline
\multicolumn{3}{|p{\descWidth}|}{{\bf Description:}   {\em m mode}} \\
\hline{\bf Range} & &  {\bf Default:} (none) \\\multicolumn{1}{|p{\maxVarWidth}|}{\centering -100:100} & \multicolumn{2}{p{\paraWidth}|}{-l to l} \\\hline
\end{tabular*}

\vspace{0.5cm}\noindent \begin{tabular*}{\tableWidth}{|c|l@{\extracolsep{\fill}}r|}
\hline
\multicolumn{1}{|p{\maxVarWidth}}{id\_power} & {\bf Scope:} private & INT \\\hline
\multicolumn{3}{|p{\descWidth}|}{{\bf Description:}   {\em related to order of polynomial in compact J pulse}} \\
\hline{\bf Range} & &  {\bf Default:} 3 \\\multicolumn{1}{|p{\maxVarWidth}|}{\centering 3:6} & \multicolumn{2}{p{\paraWidth}|}{best to choose between 3 and 6} \\\hline
\end{tabular*}

\vspace{0.5cm}\noindent \begin{tabular*}{\tableWidth}{|c|l@{\extracolsep{\fill}}r|}
\hline
\multicolumn{1}{|p{\maxVarWidth}}{jcoeff\_r0} & {\bf Scope:} private & REAL \\\hline
\multicolumn{3}{|p{\descWidth}|}{{\bf Description:}   {\em real part of constant for fitted linearized J initial data}} \\
\hline{\bf Range} & &  {\bf Default:} (none) \\\multicolumn{1}{|p{\maxVarWidth}|}{\centering *:*} & \multicolumn{2}{p{\paraWidth}|}{any value obtained from Nigel's fitting script} \\\hline
\end{tabular*}

\vspace{0.5cm}\noindent \begin{tabular*}{\tableWidth}{|c|l@{\extracolsep{\fill}}r|}
\hline
\multicolumn{1}{|p{\maxVarWidth}}{jcoeff\_r0i} & {\bf Scope:} private & REAL \\\hline
\multicolumn{3}{|p{\descWidth}|}{{\bf Description:}   {\em imag part of constant for fitted linearized J initial data}} \\
\hline{\bf Range} & &  {\bf Default:} (none) \\\multicolumn{1}{|p{\maxVarWidth}|}{\centering *:*} & \multicolumn{2}{p{\paraWidth}|}{any value obtained from Nigel's fitting script} \\\hline
\end{tabular*}

\vspace{0.5cm}\noindent \begin{tabular*}{\tableWidth}{|c|l@{\extracolsep{\fill}}r|}
\hline
\multicolumn{1}{|p{\maxVarWidth}}{jcoeff\_r0r} & {\bf Scope:} private & REAL \\\hline
\multicolumn{3}{|p{\descWidth}|}{{\bf Description:}   {\em real part of constant for fitted linearized J initial data}} \\
\hline{\bf Range} & &  {\bf Default:} (none) \\\multicolumn{1}{|p{\maxVarWidth}|}{\centering *:*} & \multicolumn{2}{p{\paraWidth}|}{any value obtained from Nigel's fitting script} \\\hline
\end{tabular*}

\vspace{0.5cm}\noindent \begin{tabular*}{\tableWidth}{|c|l@{\extracolsep{\fill}}r|}
\hline
\multicolumn{1}{|p{\maxVarWidth}}{jcoeff\_r1} & {\bf Scope:} private & REAL \\\hline
\multicolumn{3}{|p{\descWidth}|}{{\bf Description:}   {\em imag part of constant for fitted linearized J initial data}} \\
\hline{\bf Range} & &  {\bf Default:} (none) \\\multicolumn{1}{|p{\maxVarWidth}|}{\centering *:*} & \multicolumn{2}{p{\paraWidth}|}{any value obtained from Nigel's fitting script} \\\hline
\end{tabular*}

\vspace{0.5cm}\noindent \begin{tabular*}{\tableWidth}{|c|l@{\extracolsep{\fill}}r|}
\hline
\multicolumn{1}{|p{\maxVarWidth}}{jcoeff\_r1i} & {\bf Scope:} private & REAL \\\hline
\multicolumn{3}{|p{\descWidth}|}{{\bf Description:}   {\em imag part of constant for fitted linearized J initial data}} \\
\hline{\bf Range} & &  {\bf Default:} (none) \\\multicolumn{1}{|p{\maxVarWidth}|}{\centering *:*} & \multicolumn{2}{p{\paraWidth}|}{any value obtained from Nigel's fitting script} \\\hline
\end{tabular*}

\vspace{0.5cm}\noindent \begin{tabular*}{\tableWidth}{|c|l@{\extracolsep{\fill}}r|}
\hline
\multicolumn{1}{|p{\maxVarWidth}}{jcoeff\_r1r} & {\bf Scope:} private & REAL \\\hline
\multicolumn{3}{|p{\descWidth}|}{{\bf Description:}   {\em real part of constant for fitted linearized J initial data}} \\
\hline{\bf Range} & &  {\bf Default:} (none) \\\multicolumn{1}{|p{\maxVarWidth}|}{\centering *:*} & \multicolumn{2}{p{\paraWidth}|}{any value obtained from Nigel's fitting script} \\\hline
\end{tabular*}

\vspace{0.5cm}\noindent \begin{tabular*}{\tableWidth}{|c|l@{\extracolsep{\fill}}r|}
\hline
\multicolumn{1}{|p{\maxVarWidth}}{jcoeff\_r3} & {\bf Scope:} private & REAL \\\hline
\multicolumn{3}{|p{\descWidth}|}{{\bf Description:}   {\em real part of constant for fitted linearized J initial data}} \\
\hline{\bf Range} & &  {\bf Default:} (none) \\\multicolumn{1}{|p{\maxVarWidth}|}{\centering *:*} & \multicolumn{2}{p{\paraWidth}|}{any value obtained from Nigel's fitting script} \\\hline
\end{tabular*}

\vspace{0.5cm}\noindent \begin{tabular*}{\tableWidth}{|c|l@{\extracolsep{\fill}}r|}
\hline
\multicolumn{1}{|p{\maxVarWidth}}{jcoeff\_r3i} & {\bf Scope:} private & REAL \\\hline
\multicolumn{3}{|p{\descWidth}|}{{\bf Description:}   {\em imag part of constant for fitted linearized J initial data}} \\
\hline{\bf Range} & &  {\bf Default:} (none) \\\multicolumn{1}{|p{\maxVarWidth}|}{\centering *:*} & \multicolumn{2}{p{\paraWidth}|}{any value obtained from Nigel's fitting script} \\\hline
\end{tabular*}

\vspace{0.5cm}\noindent \begin{tabular*}{\tableWidth}{|c|l@{\extracolsep{\fill}}r|}
\hline
\multicolumn{1}{|p{\maxVarWidth}}{jcoeff\_r3r} & {\bf Scope:} private & REAL \\\hline
\multicolumn{3}{|p{\descWidth}|}{{\bf Description:}   {\em real part of constant for fitted linearized J initial data}} \\
\hline{\bf Range} & &  {\bf Default:} (none) \\\multicolumn{1}{|p{\maxVarWidth}|}{\centering *:*} & \multicolumn{2}{p{\paraWidth}|}{any value obtained from Nigel's fitting script} \\\hline
\end{tabular*}

\vspace{0.5cm}\noindent \begin{tabular*}{\tableWidth}{|c|l@{\extracolsep{\fill}}r|}
\hline
\multicolumn{1}{|p{\maxVarWidth}}{n\_dissip\_one\_outside\_eq} & {\bf Scope:} private & REAL \\\hline
\multicolumn{3}{|p{\descWidth}|}{{\bf Description:}   {\em determines dissipation radius rD1 = 1 + dd*N\_dissip\_one\_outside\_eq}} \\
\hline{\bf Range} & &  {\bf Default:} (none) \\\multicolumn{1}{|p{\maxVarWidth}|}{\centering *:*} & \multicolumn{2}{p{\paraWidth}|}{any value} \\\hline
\end{tabular*}

\vspace{0.5cm}\noindent \begin{tabular*}{\tableWidth}{|c|l@{\extracolsep{\fill}}r|}
\hline
\multicolumn{1}{|p{\maxVarWidth}}{n\_dissip\_zero\_outside\_eq} & {\bf Scope:} private & REAL \\\hline
\multicolumn{3}{|p{\descWidth}|}{{\bf Description:}   {\em determines dissipation radius rD0 = 1 + dd*N\_dissip\_zero\_outside\_eq}} \\
\hline{\bf Range} & &  {\bf Default:} 1 \\\multicolumn{1}{|p{\maxVarWidth}|}{\centering 0:*} & \multicolumn{2}{p{\paraWidth}|}{non-negative} \\\hline
\end{tabular*}

\vspace{0.5cm}\noindent \begin{tabular*}{\tableWidth}{|c|l@{\extracolsep{\fill}}r|}
\hline
\multicolumn{1}{|p{\maxVarWidth}}{null\_dissip} & {\bf Scope:} private & REAL \\\hline
\multicolumn{3}{|p{\descWidth}|}{{\bf Description:}   {\em dissipation factor for the radial evolution stencil}} \\
\hline{\bf Range} & &  {\bf Default:} 0.05 \\\multicolumn{1}{|p{\maxVarWidth}|}{\centering 0.0:1.0} & \multicolumn{2}{p{\paraWidth}|}{postive less than 1} \\\hline
\end{tabular*}

\vspace{0.5cm}\noindent \begin{tabular*}{\tableWidth}{|c|l@{\extracolsep{\fill}}r|}
\hline
\multicolumn{1}{|p{\maxVarWidth}}{psmax} & {\bf Scope:} private & REAL \\\hline
\multicolumn{3}{|p{\descWidth}|}{{\bf Description:}   {\em outer p bdry of CS pulse}} \\
\hline{\bf Range} & &  {\bf Default:} 1.0 \\\multicolumn{1}{|p{\maxVarWidth}|}{\centering *:*} & \multicolumn{2}{p{\paraWidth}|}{be reasonable though} \\\hline
\end{tabular*}

\vspace{0.5cm}\noindent \begin{tabular*}{\tableWidth}{|c|l@{\extracolsep{\fill}}r|}
\hline
\multicolumn{1}{|p{\maxVarWidth}}{psmin} & {\bf Scope:} private & REAL \\\hline
\multicolumn{3}{|p{\descWidth}|}{{\bf Description:}   {\em inner p bdry of CS pulse}} \\
\hline{\bf Range} & &  {\bf Default:} -1.0 \\\multicolumn{1}{|p{\maxVarWidth}|}{\centering *:*} & \multicolumn{2}{p{\paraWidth}|}{be reasonable though} \\\hline
\end{tabular*}

\vspace{0.5cm}\noindent \begin{tabular*}{\tableWidth}{|c|l@{\extracolsep{\fill}}r|}
\hline
\multicolumn{1}{|p{\maxVarWidth}}{qsmax} & {\bf Scope:} private & REAL \\\hline
\multicolumn{3}{|p{\descWidth}|}{{\bf Description:}   {\em outer q bdry of CS pulse}} \\
\hline{\bf Range} & &  {\bf Default:} 1.0 \\\multicolumn{1}{|p{\maxVarWidth}|}{\centering *:*} & \multicolumn{2}{p{\paraWidth}|}{be reasonable though} \\\hline
\end{tabular*}

\vspace{0.5cm}\noindent \begin{tabular*}{\tableWidth}{|c|l@{\extracolsep{\fill}}r|}
\hline
\multicolumn{1}{|p{\maxVarWidth}}{qsmin} & {\bf Scope:} private & REAL \\\hline
\multicolumn{3}{|p{\descWidth}|}{{\bf Description:}   {\em inner q bdry of CS pulse}} \\
\hline{\bf Range} & &  {\bf Default:} -1.0 \\\multicolumn{1}{|p{\maxVarWidth}|}{\centering *:*} & \multicolumn{2}{p{\paraWidth}|}{be reasonable though} \\\hline
\end{tabular*}

\vspace{0.5cm}\noindent \begin{tabular*}{\tableWidth}{|c|l@{\extracolsep{\fill}}r|}
\hline
\multicolumn{1}{|p{\maxVarWidth}}{use\_rsylm} & {\bf Scope:} private & BOOLEAN \\\hline
\multicolumn{3}{|p{\descWidth}|}{{\bf Description:}   {\em do we use real spherical harmonics for compact pulse initial data}} \\
\hline & & {\bf Default:} yes \\\hline
\end{tabular*}

\vspace{0.5cm}\noindent \begin{tabular*}{\tableWidth}{|c|l@{\extracolsep{\fill}}r|}
\hline
\multicolumn{1}{|p{\maxVarWidth}}{wrot} & {\bf Scope:} private & REAL \\\hline
\multicolumn{3}{|p{\descWidth}|}{{\bf Description:}   {\em frequency of CS pule}} \\
\hline{\bf Range} & &  {\bf Default:} 7.5 \\\multicolumn{1}{|p{\maxVarWidth}|}{\centering *:*} & \multicolumn{2}{p{\paraWidth}|}{be reasonable though} \\\hline
\end{tabular*}

\vspace{0.5cm}\noindent \begin{tabular*}{\tableWidth}{|c|l@{\extracolsep{\fill}}r|}
\hline
\multicolumn{1}{|p{\maxVarWidth}}{xmax} & {\bf Scope:} private & REAL \\\hline
\multicolumn{3}{|p{\descWidth}|}{{\bf Description:}   {\em right boundary of CS pulse}} \\
\hline{\bf Range} & &  {\bf Default:} .8 \\\multicolumn{1}{|p{\maxVarWidth}|}{\centering .5:1} & \multicolumn{2}{p{\paraWidth}|}{choose  .5 {\textless} xmin {\textless} xmax {\textless} 1  } \\\hline
\end{tabular*}

\vspace{0.5cm}\noindent \begin{tabular*}{\tableWidth}{|c|l@{\extracolsep{\fill}}r|}
\hline
\multicolumn{1}{|p{\maxVarWidth}}{xmin} & {\bf Scope:} private & REAL \\\hline
\multicolumn{3}{|p{\descWidth}|}{{\bf Description:}   {\em left boundary of CS pulse}} \\
\hline{\bf Range} & &  {\bf Default:} .56 \\\multicolumn{1}{|p{\maxVarWidth}|}{\centering .5:1} & \multicolumn{2}{p{\paraWidth}|}{choose  .5 {\textless} xmin {\textless} xmax {\textless} 1  } \\\hline
\end{tabular*}

\vspace{0.5cm}\noindent \begin{tabular*}{\tableWidth}{|c|l@{\extracolsep{\fill}}r|}
\hline
\multicolumn{1}{|p{\maxVarWidth}}{boundary\_data} & {\bf Scope:} restricted & KEYWORD \\\hline
\multicolumn{3}{|p{\descWidth}|}{{\bf Description:}   {\em Choose boundary data type}} \\
\hline{\bf Range} & &  {\bf Default:} flat \\\multicolumn{1}{|p{\maxVarWidth}|}{\centering whitehole} & \multicolumn{2}{p{\paraWidth}|}{White hole boundary data} \\\multicolumn{1}{|p{\maxVarWidth}|}{\centering flat} & \multicolumn{2}{p{\paraWidth}|}{flat boundary data} \\\multicolumn{1}{|p{\maxVarWidth}|}{\centering randomJ} & \multicolumn{2}{p{\paraWidth}|}{random number boundary data} \\\hline
\end{tabular*}

\vspace{0.5cm}\noindent \begin{tabular*}{\tableWidth}{|c|l@{\extracolsep{\fill}}r|}
\hline
\multicolumn{1}{|p{\maxVarWidth}}{debug\_skip\_b\_update} & {\bf Scope:} restricted & BOOLEAN \\\hline
\multicolumn{3}{|p{\descWidth}|}{{\bf Description:}   {\em Should the update of B be turned off?}} \\
\hline & & {\bf Default:} no \\\hline
\end{tabular*}

\vspace{0.5cm}\noindent \begin{tabular*}{\tableWidth}{|c|l@{\extracolsep{\fill}}r|}
\hline
\multicolumn{1}{|p{\maxVarWidth}}{debug\_skip\_cb\_update} & {\bf Scope:} restricted & BOOLEAN \\\hline
\multicolumn{3}{|p{\descWidth}|}{{\bf Description:}   {\em Should the update of CB be turned off?}} \\
\hline & & {\bf Default:} no \\\hline
\end{tabular*}

\vspace{0.5cm}\noindent \begin{tabular*}{\tableWidth}{|c|l@{\extracolsep{\fill}}r|}
\hline
\multicolumn{1}{|p{\maxVarWidth}}{debug\_skip\_ck\_update} & {\bf Scope:} restricted & BOOLEAN \\\hline
\multicolumn{3}{|p{\descWidth}|}{{\bf Description:}   {\em Should the update of CK be turned off?}} \\
\hline & & {\bf Default:} no \\\hline
\end{tabular*}

\vspace{0.5cm}\noindent \begin{tabular*}{\tableWidth}{|c|l@{\extracolsep{\fill}}r|}
\hline
\multicolumn{1}{|p{\maxVarWidth}}{debug\_skip\_evolution} & {\bf Scope:} restricted & BOOLEAN \\\hline
\multicolumn{3}{|p{\descWidth}|}{{\bf Description:}   {\em Should the evolution be turned off?}} \\
\hline & & {\bf Default:} no \\\hline
\end{tabular*}

\vspace{0.5cm}\noindent \begin{tabular*}{\tableWidth}{|c|l@{\extracolsep{\fill}}r|}
\hline
\multicolumn{1}{|p{\maxVarWidth}}{debug\_skip\_j\_update} & {\bf Scope:} restricted & BOOLEAN \\\hline
\multicolumn{3}{|p{\descWidth}|}{{\bf Description:}   {\em Should the update of J be turned off?}} \\
\hline & & {\bf Default:} no \\\hline
\end{tabular*}

\vspace{0.5cm}\noindent \begin{tabular*}{\tableWidth}{|c|l@{\extracolsep{\fill}}r|}
\hline
\multicolumn{1}{|p{\maxVarWidth}}{debug\_skip\_nu\_update} & {\bf Scope:} restricted & BOOLEAN \\\hline
\multicolumn{3}{|p{\descWidth}|}{{\bf Description:}   {\em Should the update of NU be turned off?}} \\
\hline & & {\bf Default:} no \\\hline
\end{tabular*}

\vspace{0.5cm}\noindent \begin{tabular*}{\tableWidth}{|c|l@{\extracolsep{\fill}}r|}
\hline
\multicolumn{1}{|p{\maxVarWidth}}{debug\_skip\_q\_update} & {\bf Scope:} restricted & BOOLEAN \\\hline
\multicolumn{3}{|p{\descWidth}|}{{\bf Description:}   {\em Should the update of Q be turned off?}} \\
\hline & & {\bf Default:} no \\\hline
\end{tabular*}

\vspace{0.5cm}\noindent \begin{tabular*}{\tableWidth}{|c|l@{\extracolsep{\fill}}r|}
\hline
\multicolumn{1}{|p{\maxVarWidth}}{debug\_skip\_u\_update} & {\bf Scope:} restricted & BOOLEAN \\\hline
\multicolumn{3}{|p{\descWidth}|}{{\bf Description:}   {\em Should the update of U be turned off?}} \\
\hline & & {\bf Default:} no \\\hline
\end{tabular*}

\vspace{0.5cm}\noindent \begin{tabular*}{\tableWidth}{|c|l@{\extracolsep{\fill}}r|}
\hline
\multicolumn{1}{|p{\maxVarWidth}}{debug\_skip\_w\_update} & {\bf Scope:} restricted & BOOLEAN \\\hline
\multicolumn{3}{|p{\descWidth}|}{{\bf Description:}   {\em Should the update of W be turned off?}} \\
\hline & & {\bf Default:} no \\\hline
\end{tabular*}

\vspace{0.5cm}\noindent \begin{tabular*}{\tableWidth}{|c|l@{\extracolsep{\fill}}r|}
\hline
\multicolumn{1}{|p{\maxVarWidth}}{debug\_verbose} & {\bf Scope:} restricted & BOOLEAN \\\hline
\multicolumn{3}{|p{\descWidth}|}{{\bf Description:}   {\em should debugging messages be printed?}} \\
\hline & & {\bf Default:} no \\\hline
\end{tabular*}

\vspace{0.5cm}\noindent \begin{tabular*}{\tableWidth}{|c|l@{\extracolsep{\fill}}r|}
\hline
\multicolumn{1}{|p{\maxVarWidth}}{diagnostics\_coord\_x} & {\bf Scope:} restricted & REAL \\\hline
\multicolumn{3}{|p{\descWidth}|}{{\bf Description:}   {\em the coordinate x at which diagnostics are to be measured}} \\
\hline{\bf Range} & &  {\bf Default:} 0.0 \\\multicolumn{1}{|p{\maxVarWidth}|}{\centering 0:1} & \multicolumn{2}{p{\paraWidth}|}{will test only for x {\textgreater}= NullGrid::null\_xin.} \\\hline
\end{tabular*}

\vspace{0.5cm}\noindent \begin{tabular*}{\tableWidth}{|c|l@{\extracolsep{\fill}}r|}
\hline
\multicolumn{1}{|p{\maxVarWidth}}{first\_order\_scheme} & {\bf Scope:} restricted & BOOLEAN \\\hline
\multicolumn{3}{|p{\descWidth}|}{{\bf Description:}   {\em Should the first order (angular) scheme be used?}} \\
\hline & & {\bf Default:} yes \\\hline
\end{tabular*}

\vspace{0.5cm}\noindent \begin{tabular*}{\tableWidth}{|c|l@{\extracolsep{\fill}}r|}
\hline
\multicolumn{1}{|p{\maxVarWidth}}{initial\_j\_data} & {\bf Scope:} restricted & KEYWORD \\\hline
\multicolumn{3}{|p{\descWidth}|}{{\bf Description:}   {\em What kind of initial data for J shall we pick?}} \\
\hline{\bf Range} & &  {\bf Default:} vanishing\_J\_scri \\\multicolumn{1}{|p{\maxVarWidth}|}{\centering vanishing\_J} & \multicolumn{2}{p{\paraWidth}|}{vanishing J} \\\multicolumn{1}{|p{\maxVarWidth}|}{\centering vanishing\_J\_scri} & \multicolumn{2}{p{\paraWidth}|}{vanishing J at scri} \\\multicolumn{1}{|p{\maxVarWidth}|}{\centering rotating\_pulse} & \multicolumn{2}{p{\paraWidth}|}{rotating pulse} \\\multicolumn{1}{|p{\maxVarWidth}|}{\centering compact\_J} & \multicolumn{2}{p{\paraWidth}|}{compact suport J} \\\multicolumn{1}{|p{\maxVarWidth}|}{\centering smooth\_J} & \multicolumn{2}{p{\paraWidth}|}{smoothly vanishing J} \\\multicolumn{1}{|p{\maxVarWidth}|}{\centering extracted\_J} & \multicolumn{2}{p{\paraWidth}|}{J as given by extraction} \\\multicolumn{1}{|p{\maxVarWidth}|}{\centering polynomial\_J} & \multicolumn{2}{p{\paraWidth}|}{J as third power polynomial to vanish J\_xx at scri} \\\multicolumn{1}{|p{\maxVarWidth}|}{see [1] below} & \multicolumn{2}{p{\paraWidth}|}{linearized fitted J based on some previous run} \\\hline
\end{tabular*}

\vspace{0.5cm}\noindent {\bf [1]} \noindent \begin{verbatim}fitted\_linearized\_J\end{verbatim}\noindent \begin{tabular*}{\tableWidth}{|c|l@{\extracolsep{\fill}}r|}
\hline
\multicolumn{1}{|p{\maxVarWidth}}{old\_j\_xderiv} & {\bf Scope:} restricted & BOOLEAN \\\hline
\multicolumn{3}{|p{\descWidth}|}{{\bf Description:}   {\em should we compute the x derivative of J with the old values?}} \\
\hline & & {\bf Default:} no \\\hline
\end{tabular*}

\vspace{0.5cm}\noindent \begin{tabular*}{\tableWidth}{|c|l@{\extracolsep{\fill}}r|}
\hline
\multicolumn{1}{|p{\maxVarWidth}}{timestep\_method} & {\bf Scope:} shared from TIME & KEYWORD \\\hline
\end{tabular*}

\vspace{0.5cm}\parskip = 10pt 

\section{Interfaces} 


\parskip = 0pt

\vspace{3mm} \subsection*{General}

\noindent {\bf Implements}: 

nullevolve
\vspace{2mm}

\noindent {\bf Inherits}: 

nullinterp

nullgrid

nullvars

time
\vspace{2mm}
\subsection*{Grid Variables}
\vspace{5mm}\subsubsection{PRIVATE GROUPS}

\vspace{5mm}

\begin{tabular*}{150mm}{|c|c@{\extracolsep{\fill}}|rl|} \hline 
~ {\bf Group Names} ~ & ~ {\bf Variable Names} ~  &{\bf Details} ~ & ~\\ 
\hline 
distmp &  & compact & 0 \\ 
 & distmp\_F & description & temporaries used for dissipation \\ 
 & distmp\_d2F & dimensions & 2 \\ 
 & distmp\_d4F & distribution & DEFAULT \\ 
 &  & ghostsize & NULLGRID::N\_ANG\_GHOST\_PTS \\ 
& ~ & ghostsize & NULLGRID::N\_ANG\_GHOST\_PTS \\ 
 &  & group type & ARRAY \\ 
 &  & size & (NULLGRID::N\_ANG\_PTS\_INSIDE\_EQ+2*(NULLGRID::N\_ANG\_EV\_OUTSIDE\_EQ+NULLGRID::N\_ANG\_STENCIL\_SIZE)) \\ 
& ~ & size & (NULLGRID::N\_ANG\_PTS\_INSIDE\_EQ+2*(NULLGRID::N\_ANG\_EV\_OUTSIDE\_EQ+NULLGRID::N\_ANG\_STENCIL\_SIZE)) \\ 
 &  & timelevels & 1 \\ 
 &  & vararray\_size & 2 \\ 
 &  & variable type & COMPLEX \\ 
\hline 
dissip\_mask & dissip\_mask & compact & 0 \\ 
 &  & description & dissipation mask \\ 
 &  & dimensions & 2 \\ 
 &  & distribution & DEFAULT \\ 
 &  & ghostsize & NULLGRID::N\_ANG\_GHOST\_PTS \\ 
& ~ & ghostsize & NULLGRID::N\_ANG\_GHOST\_PTS \\ 
 &  & group type & ARRAY \\ 
 &  & size & (NULLGRID::N\_ANG\_PTS\_INSIDE\_EQ+2*(NULLGRID::N\_ANG\_EV\_OUTSIDE\_EQ+NULLGRID::N\_ANG\_STENCIL\_SIZE)) \\ 
& ~ & size & (NULLGRID::N\_ANG\_PTS\_INSIDE\_EQ+2*(NULLGRID::N\_ANG\_EV\_OUTSIDE\_EQ+NULLGRID::N\_ANG\_STENCIL\_SIZE)) \\ 
 &  & timelevels & 1 \\ 
 &  & variable type & REAL \\ 
\hline 
diagtmp & diagtmp & compact & 0 \\ 
 &  & description & temporaries used for diagnostics \\ 
 &  & dimensions & 2 \\ 
 &  & distribution & DEFAULT \\ 
 &  & ghostsize & NULLGRID::N\_ANG\_GHOST\_PTS \\ 
& ~ & ghostsize & NULLGRID::N\_ANG\_GHOST\_PTS \\ 
 &  & group type & ARRAY \\ 
 &  & size & (NULLGRID::N\_ANG\_PTS\_INSIDE\_EQ+2*(NULLGRID::N\_ANG\_EV\_OUTSIDE\_EQ+NULLGRID::N\_ANG\_STENCIL\_SIZE)) \\ 
& ~ & size & (NULLGRID::N\_ANG\_PTS\_INSIDE\_EQ+2*(NULLGRID::N\_ANG\_EV\_OUTSIDE\_EQ+NULLGRID::N\_ANG\_STENCIL\_SIZE)) \\ 
 &  & timelevels & 1 \\ 
 &  & vararray\_size & 2 \\ 
 &  & variable type & COMPLEX \\ 
\hline 
aux\_mask &  & compact & 0 \\ 
 & auxiliary\_maskn & description & auxiliary evolution masks \\ 
 & auxiliary\_masks & dimensions & 2 \\ 
 &  & distribution & DEFAULT \\ 
 &  & ghostsize & NULLGRID::N\_ANG\_GHOST\_PTS \\ 
& ~ & ghostsize & NULLGRID::N\_ANG\_GHOST\_PTS \\ 
 &  & group type & ARRAY \\ 
 &  & size & (NULLGRID::N\_ANG\_PTS\_INSIDE\_EQ+2*(NULLGRID::N\_ANG\_EV\_OUTSIDE\_EQ+NULLGRID::N\_ANG\_STENCIL\_SIZE)) \\ 
& ~ & size & (NULLGRID::N\_ANG\_PTS\_INSIDE\_EQ+2*(NULLGRID::N\_ANG\_EV\_OUTSIDE\_EQ+NULLGRID::N\_ANG\_STENCIL\_SIZE)) \\ 
 &  & timelevels & 1 \\ 
 &  & variable type & INT \\ 
\hline 
eth4\_mask &  & compact & 0 \\ 
 & eth4\_maskn & description & dissipation masks \\ 
& ~ & description &  to protect the startup from taking eth4 \\ 
 & eth4\_masks & dimensions & 2 \\ 
 &  & distribution & DEFAULT \\ 
 &  & ghostsize & NULLGRID::N\_ANG\_GHOST\_PTS \\ 
& ~ & ghostsize & NULLGRID::N\_ANG\_GHOST\_PTS \\ 
 &  & group type & ARRAY \\ 
 &  & size & (NULLGRID::N\_ANG\_PTS\_INSIDE\_EQ+2*(NULLGRID::N\_ANG\_EV\_OUTSIDE\_EQ+NULLGRID::N\_ANG\_STENCIL\_SIZE)) \\ 
& ~ & size & (NULLGRID::N\_ANG\_PTS\_INSIDE\_EQ+2*(NULLGRID::N\_ANG\_EV\_OUTSIDE\_EQ+NULLGRID::N\_ANG\_STENCIL\_SIZE)) \\ 
 &  & timelevels & 1 \\ 
 &  & variable type & INT \\ 
\hline 
jrad &  & compact & 0 \\ 
 & jcn\_rad & dimensions & 1 \\ 
 & jcs\_rad & distribution & CONSTANT \\ 
 &  & group type & ARRAY \\ 
 &  & size & NULLGRID::N\_RADIAL\_PTS \\ 
 &  & timelevels & 1 \\ 
 &  & variable type & COMPLEX \\ 
\hline 
\end{tabular*} 



\vspace{5mm}
\vspace{5mm}

\begin{tabular*}{150mm}{|c|c@{\extracolsep{\fill}}|rl|} \hline 
~ {\bf Group Names} ~ & ~ {\bf Variable Names} ~  &{\bf Details} ~ & ~ \\ 
\hline 
dxjrad &  & compact & 0 \\ 
 & dxjcn\_rad & dimensions & 1 \\ 
 & dxjcs\_rad & distribution & CONSTANT \\ 
 &  & group type & ARRAY \\ 
 &  & size & NULLGRID::N\_RADIAL\_PTS \\ 
 &  & timelevels & 1 \\ 
 &  & variable type & COMPLEX \\ 
\hline 
\end{tabular*} 



\vspace{5mm}\parskip = 10pt 

\section{Schedule} 


\parskip = 0pt


\noindent This section lists all the variables which are assigned storage by thorn PITTNullCode/NullEvolve.  Storage can either last for the duration of the run ({\bf Always} means that if this thorn is activated storage will be assigned, {\bf Conditional} means that if this thorn is activated storage will be assigned for the duration of the run if some condition is met), or can be turned on for the duration of a schedule function.


\subsection*{Storage}

\hspace{5mm}

 \begin{tabular*}{160mm}{ll} 

{\bf Always:}& {\bf Conditional:} \\ 
 NullVars::realcharfuncs[2] &  NullVars::cmplxcharfuncs\_aux[2]\\ 
 NullVars::cmplxcharfuncs\_basic[2] & ~\\ 
 NullVars::null\_mask & ~\\ 
 NullGrid::EG\_mask NullGrid::EQ\_mask NullGrid::EV\_mask & ~\\ 
 eth4\_mask dissip\_mask & ~\\ 
 diagtmp aux\_mask & ~\\ 
 Jrad dxJrad & ~\\ 
~ & ~\\ 
\end{tabular*} 


\subsection*{Scheduled Functions}
\vspace{5mm}

\noindent {\bf CCTK\_POSTINITIAL}   (conditional) 

\hspace{5mm} nullevol\_initial 

\hspace{5mm}{\it null init data } 


\hspace{5mm}

 \begin{tabular*}{160mm}{cll} 
~ & After:  & harmidata\_init\_to\_adm \\ 
~& ~ &harmidata\_pinit\_to\_adm\\ 
~& ~ &mol\_fillalllevels\\ 
~& ~ &adm\_bssn\_init\\ 
~ & Options:  & global \\ 
~ & Type:  & group \\ 
\end{tabular*} 


\vspace{5mm}

\noindent {\bf CCTK\_INITIAL}   (conditional) 

\hspace{5mm} nullevol\_initializearrays 

\hspace{5mm}{\it initialize all arrays to large values } 


\hspace{5mm}

 \begin{tabular*}{160mm}{cll} 
~ & Before:  & nullevol\_initial \\ 
~ & Language:  & fortran \\ 
~ & Type:  & function \\ 
\end{tabular*} 


\vspace{5mm}

\noindent {\bf NullEvol\_BoundaryInit}   (conditional) 

\hspace{5mm} nullevol\_bdry\_flat 

\hspace{5mm}{\it give flat boundary conditions for the null metric } 


\hspace{5mm}

 \begin{tabular*}{160mm}{cll} 
~ & Language:  & fortran \\ 
~ & Options:  & global \\ 
~ & Type:  & function \\ 
\end{tabular*} 


\vspace{5mm}

\noindent {\bf NullEvol\_BoundaryInit}   (conditional) 

\hspace{5mm} nullevol\_bdry\_whitehole 

\hspace{5mm}{\it give white hole boundary conditions for the null metric } 


\hspace{5mm}

 \begin{tabular*}{160mm}{cll} 
~ & Language:  & fortran \\ 
~ & Options:  & global \\ 
~ & Type:  & function \\ 
\end{tabular*} 


\vspace{5mm}

\noindent {\bf NullEvol\_BoundaryInit}   (conditional) 

\hspace{5mm} nullevol\_bdry\_randomj 

\hspace{5mm}{\it give random j boundary conditions for the null metric } 


\hspace{5mm}

 \begin{tabular*}{160mm}{cll} 
~ & Language:  & fortran \\ 
~ & Options:  & global \\ 
~ & Type:  & function \\ 
\end{tabular*} 


\vspace{5mm}

\noindent {\bf NullEvol\_Boundary}   (conditional) 

\hspace{5mm} nullevol\_bdry\_flat 

\hspace{5mm}{\it give flat boundary conditions for the null metric } 


\hspace{5mm}

 \begin{tabular*}{160mm}{cll} 
~ & Language:  & fortran \\ 
~ & Options:  & global \\ 
~ & Type:  & function \\ 
\end{tabular*} 


\vspace{5mm}

\noindent {\bf NullEvol\_Boundary}   (conditional) 

\hspace{5mm} nullevol\_bdry\_whitehole 

\hspace{5mm}{\it give white hole boundary conditions for the null metric } 


\hspace{5mm}

 \begin{tabular*}{160mm}{cll} 
~ & Language:  & fortran \\ 
~ & Options:  & global \\ 
~ & Type:  & function \\ 
\end{tabular*} 


\vspace{5mm}

\noindent {\bf NullEvol\_Boundary}   (conditional) 

\hspace{5mm} nullevol\_bdry\_randomj 

\hspace{5mm}{\it give random j boundary conditions for the null metric } 


\hspace{5mm}

 \begin{tabular*}{160mm}{cll} 
~ & Language:  & fortran \\ 
~ & Options:  & global \\ 
~ & Type:  & function \\ 
\end{tabular*} 


\vspace{5mm}

\noindent {\bf CCTK\_EVOL}   (conditional) 

\hspace{5mm} nullevol\_resettop 

\hspace{5mm}{\it reset top values } 


\hspace{5mm}

 \begin{tabular*}{160mm}{cll} 
~ & Before:  & nullevol\_boundary \\ 
~ & Language:  & fortran \\ 
~ & Options:  & global \\ 
~ & Type:  & function \\ 
\end{tabular*} 


\vspace{5mm}

\noindent {\bf CCTK\_EVOL}   (conditional) 

\hspace{5mm} nullevol\_boundary 

\hspace{5mm}{\it boundary data } 


\hspace{5mm}

 \begin{tabular*}{160mm}{cll} 
~ & After:  & harmevol\_to\_adm \\ 
~& ~ &mol\_evolution\\ 
~ & Options:  & global \\ 
~ & Type:  & group \\ 
\end{tabular*} 


\vspace{5mm}

\noindent {\bf CCTK\_EVOL}   (conditional) 

\hspace{5mm} nullevol\_step 

\hspace{5mm}{\it evolution } 


\hspace{5mm}

 \begin{tabular*}{160mm}{cll} 
~ & After:  & nullevol\_boundary \\ 
~ & Language:  & fortran \\ 
~ & Options:  & global \\ 
~ & Storage:  & distmp \\ 
~ & Type:  & function \\ 
\end{tabular*} 


\vspace{5mm}

\noindent {\bf NullEvol\_Initial}   (conditional) 

\hspace{5mm} nullevol\_initialdata 

\hspace{5mm}{\it give j on the initial null hypersurface } 


\hspace{5mm}

 \begin{tabular*}{160mm}{cll} 
~ & After:  & nullevol\_boundaryinit \\ 
~ & Language:  & fortran \\ 
~ & Options:  & global \\ 
~ & Type:  & function \\ 
\end{tabular*} 


\vspace{5mm}

\noindent {\bf NullEvol\_Initial}   (conditional) 

\hspace{5mm} nullevol\_boundaryinit 

\hspace{5mm}{\it boundary data for the characteristic data } 


\hspace{5mm}

 \begin{tabular*}{160mm}{cll} 
~ & Before:  & nullevol\_initialdata \\ 
~& ~ &nullevol\_initialslice\\ 
~ & Type:  & group \\ 
\end{tabular*} 


\vspace{5mm}

\noindent {\bf NullEvol\_Initial}   (conditional) 

\hspace{5mm} nullevol\_initialslice 

\hspace{5mm}{\it construct null metric on the initial null hypersurface } 


\hspace{5mm}

 \begin{tabular*}{160mm}{cll} 
~ & After:  & nullevol\_boundaryinit \\ 
~& ~ &nullevol\_initialdata\\ 
~ & Language:  & fortran \\ 
~ & Options:  & global \\ 
~ & Type:  & function \\ 
\end{tabular*} 


\vspace{5mm}

\noindent {\bf NullEvol\_Initial}   (conditional) 

\hspace{5mm} nullevol\_diag 

\hspace{5mm}{\it diagnostics of the characteristic code } 


\hspace{5mm}

 \begin{tabular*}{160mm}{cll} 
~ & After:  & nullevol\_initialslice \\ 
~ & Language:  & fortran \\ 
~ & Options:  & global \\ 
~ & Storage:  & diagtmp \\ 
~ & Type:  & function \\ 
\end{tabular*} 


\vspace{5mm}

\noindent {\bf CCTK\_EVOL}   (conditional) 

\hspace{5mm} nullevol\_diag 

\hspace{5mm}{\it diagnostics of the characteristic code } 


\hspace{5mm}

 \begin{tabular*}{160mm}{cll} 
~ & After:  & nullevol\_step \\ 
~ & Language:  & fortran \\ 
~ & Options:  & global \\ 
~ & Storage:  & diagtmp \\ 
~ & Type:  & function \\ 
\end{tabular*} 



\vspace{5mm}\parskip = 10pt 
\end{document}
