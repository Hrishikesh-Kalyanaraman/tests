% *======================================================================*
%  Cactus Thorn template for ThornGuide documentation
%  Author: Ian Kelley
%  Date: Sun Jun 02, 2002
%
%  Thorn documentation in the latex file doc/documentation.tex
%  will be included in ThornGuides built with the Cactus make system.
%  The scripts employed by the make system automatically include
%  pages about variables, parameters and scheduling parsed from the
%  relevant thorn CCL files.
%
%  This template contains guidelines which help to assure that your
%  documentation will be correctly added to ThornGuides. More
%  information is available in the Cactus UsersGuide.
%
%  Guidelines:
%   - Do not change anything before the line
%       % START CACTUS THORNGUIDE",
%     except for filling in the title, author, date, etc. fields.
%        - Each of these fields should only be on ONE line.
%        - Author names should be separated with a \\ or a comma.
%   - You can define your own macros, but they must appear after
%     the START CACTUS THORNGUIDE line, and must not redefine standard
%     latex commands.
%   - To avoid name clashes with other thorns, 'labels', 'citations',
%     'references', and 'image' names should conform to the following
%     convention:
%       ARRANGEMENT_THORN_LABEL
%     For example, an image wave.eps in the arrangement CactusWave and
%     thorn WaveToyC should be renamed to CactusWave_WaveToyC_wave.eps
%   - Graphics should only be included using the graphicx package.
%     More specifically, with the "\includegraphics" command.  Do
%     not specify any graphic file extensions in your .tex file. This
%     will allow us to create a PDF version of the ThornGuide
%     via pdflatex.
%   - References should be included with the latex "\bibitem" command.
%   - Use \begin{abstract}...\end{abstract} instead of \abstract{...}
%   - Do not use \appendix, instead include any appendices you need as
%     standard sections.
%   - For the benefit of our Perl scripts, and for future extensions,
%     please use simple latex.
%
% *======================================================================*
%
% Example of including a graphic image:
%    \begin{figure}[ht]
% 	\begin{center}
%    	   \includegraphics[width=6cm]{/home/runner/work/tests/tests/arrangements/EinsteinAnalysis/Hydro_Analysis/doc/MyArrangement_MyThorn_MyFigure}
% 	\end{center}
% 	\caption{Illustration of this and that}
% 	\label{MyArrangement_MyThorn_MyLabel}
%    \end{figure}
%
% Example of using a label:
%   \label{MyArrangement_MyThorn_MyLabel}
%
% Example of a citation:
%    \cite{MyArrangement_MyThorn_Author99}
%
% Example of including a reference
%   \bibitem{MyArrangement_MyThorn_Author99}
%   {J. Author, {\em The Title of the Book, Journal, or periodical}, 1 (1999),
%   1--16. \texttt{http://www.nowhere.com/}}
%
% *======================================================================*

\documentclass{article}

% Use the Cactus ThornGuide style file
% (Automatically used from Cactus distribution, if you have a
%  thorn without the Cactus Flesh download this from the Cactus
%  homepage at www.cactuscode.org)
\usepackage{../../../../../doc/latex/cactus}

\newlength{\tableWidth} \newlength{\maxVarWidth} \newlength{\paraWidth} \newlength{\descWidth} \begin{document}

\title{Hydro\_Analysis}
\author{Frank Löffler}

\maketitle

% Do not delete next line
% START CACTUS THORNGUIDE

% Add all definitions used in this documentation here
%   \def\mydef etc
\def\ie{i.e.\hbox{}}
\def\del{\nabla}

% get size/spacing of "++" right, cf online C++ FAQ question 35.1
\def\Cplusplus{\hbox{C\raise.25ex\hbox{\footnotesize ++}}}


\begin{abstract}
Provides basic hydro analysis routines and quantities.
\end{abstract}


\section{Purpose}

% Do not delete next line
% END CACTUS THORNGUIDE


\section{Parameters} 


\parskip = 0pt

\setlength{\tableWidth}{160mm}

\setlength{\paraWidth}{\tableWidth}
\setlength{\descWidth}{\tableWidth}
\settowidth{\maxVarWidth}{hydro\_analysis\_rho\_max\_loc\_use\_rotatingsymmetry180}

\addtolength{\paraWidth}{-\maxVarWidth}
\addtolength{\paraWidth}{-\columnsep}
\addtolength{\paraWidth}{-\columnsep}
\addtolength{\paraWidth}{-\columnsep}

\addtolength{\descWidth}{-\columnsep}
\addtolength{\descWidth}{-\columnsep}
\addtolength{\descWidth}{-\columnsep}
\noindent \begin{tabular*}{\tableWidth}{|c|l@{\extracolsep{\fill}}r|}
\hline
\multicolumn{1}{|p{\maxVarWidth}}{hydro\_analysis\_average\_multiple\_maxima\_locations} & {\bf Scope:} private & BOOLEAN \\\hline
\multicolumn{3}{|p{\descWidth}|}{{\bf Description:}   {\em when finding more than one global maximum location, average position and use result}} \\
\hline & & {\bf Default:} false \\\hline
\end{tabular*}

\vspace{0.5cm}\noindent \begin{tabular*}{\tableWidth}{|c|l@{\extracolsep{\fill}}r|}
\hline
\multicolumn{1}{|p{\maxVarWidth}}{hydro\_analysis\_comp\_core\_rho\_centroid} & {\bf Scope:} private & BOOLEAN \\\hline
\multicolumn{3}{|p{\descWidth}|}{{\bf Description:}   {\em compute location of the centroid of rho*x in region r\_core around densest point}} \\
\hline & & {\bf Default:} no \\\hline
\end{tabular*}

\vspace{0.5cm}\noindent \begin{tabular*}{\tableWidth}{|c|l@{\extracolsep{\fill}}r|}
\hline
\multicolumn{1}{|p{\maxVarWidth}}{hydro\_analysis\_comp\_rho\_max} & {\bf Scope:} private & BOOLEAN \\\hline
\multicolumn{3}{|p{\descWidth}|}{{\bf Description:}   {\em Look for the value and location of the maximum of rho}} \\
\hline & & {\bf Default:} false \\\hline
\end{tabular*}

\vspace{0.5cm}\noindent \begin{tabular*}{\tableWidth}{|c|l@{\extracolsep{\fill}}r|}
\hline
\multicolumn{1}{|p{\maxVarWidth}}{hydro\_analysis\_comp\_rho\_max\_every} & {\bf Scope:} private & INT \\\hline
\multicolumn{3}{|p{\descWidth}|}{{\bf Description:}   {\em How often to look  for the value and location of the maximum of rho}} \\
\hline{\bf Range} & &  {\bf Default:} 1 \\\multicolumn{1}{|p{\maxVarWidth}|}{\centering 0:0} & \multicolumn{2}{p{\paraWidth}|}{Never} \\\multicolumn{1}{|p{\maxVarWidth}|}{\centering 1:*} & \multicolumn{2}{p{\paraWidth}|}{every so often} \\\hline
\end{tabular*}

\vspace{0.5cm}\noindent \begin{tabular*}{\tableWidth}{|c|l@{\extracolsep{\fill}}r|}
\hline
\multicolumn{1}{|p{\maxVarWidth}}{hydro\_analysis\_comp\_rho\_max\_origin\_distance} & {\bf Scope:} private & BOOLEAN \\\hline
\multicolumn{3}{|p{\descWidth}|}{{\bf Description:}   {\em Look for the proper distance between the maximum of the density and the origin (along a straight coordinate line)}} \\
\hline & & {\bf Default:} false \\\hline
\end{tabular*}

\vspace{0.5cm}\noindent \begin{tabular*}{\tableWidth}{|c|l@{\extracolsep{\fill}}r|}
\hline
\multicolumn{1}{|p{\maxVarWidth}}{hydro\_analysis\_comp\_vol\_weighted\_center\_of\_mass} & {\bf Scope:} private & BOOLEAN \\\hline
\multicolumn{3}{|p{\descWidth}|}{{\bf Description:}   {\em Look for the location of the volume-weighted center of mass}} \\
\hline & & {\bf Default:} false \\\hline
\end{tabular*}

\vspace{0.5cm}\noindent \begin{tabular*}{\tableWidth}{|c|l@{\extracolsep{\fill}}r|}
\hline
\multicolumn{1}{|p{\maxVarWidth}}{hydro\_analysis\_core\_rho\_rel\_min} & {\bf Scope:} private & REAL \\\hline
\multicolumn{3}{|p{\descWidth}|}{{\bf Description:}   {\em only include points where rho{\textgreater}rho\_rel\_min*rho\_max when computing centroid of rho*x around densest point}} \\
\hline{\bf Range} & &  {\bf Default:} 0.01 \\\multicolumn{1}{|p{\maxVarWidth}|}{\centering 0:*} & \multicolumn{2}{p{\paraWidth}|}{any positive value. For best results should be such that the region selected is smaller than r\_core. 1/10 of rho\_max should be fine.} \\\hline
\end{tabular*}

\vspace{0.5cm}\noindent \begin{tabular*}{\tableWidth}{|c|l@{\extracolsep{\fill}}r|}
\hline
\multicolumn{1}{|p{\maxVarWidth}}{hydro\_analysis\_interpolator\_coordinates} & {\bf Scope:} private & STRING \\\hline
\multicolumn{3}{|p{\descWidth}|}{{\bf Description:}   {\em Coordinate system}} \\
\hline{\bf Range} & &  {\bf Default:} cart3d \\\multicolumn{1}{|p{\maxVarWidth}|}{\centering .*} & \multicolumn{2}{p{\paraWidth}|}{must be a registered coordinate system} \\\hline
\end{tabular*}

\vspace{0.5cm}\noindent \begin{tabular*}{\tableWidth}{|c|l@{\extracolsep{\fill}}r|}
\hline
\multicolumn{1}{|p{\maxVarWidth}}{hydro\_analysis\_interpolator\_name} & {\bf Scope:} private & STRING \\\hline
\multicolumn{3}{|p{\descWidth}|}{{\bf Description:}   {\em Name of the interpolator}} \\
\hline{\bf Range} & &  {\bf Default:} uniform cartesian \\\multicolumn{1}{|p{\maxVarWidth}|}{see [1] below} & \multicolumn{2}{p{\paraWidth}|}{from AEILocalInterp} \\\multicolumn{1}{|p{\maxVarWidth}|}{see [1] below} & \multicolumn{2}{p{\paraWidth}|}{from AEILocalInterp} \\\multicolumn{1}{|p{\maxVarWidth}|}{see [1] below} & \multicolumn{2}{p{\paraWidth}|}{from AEILocalInterp} \\\multicolumn{1}{|p{\maxVarWidth}|}{\centering uniform cartesian} & \multicolumn{2}{p{\paraWidth}|}{from LocalInterp} \\\multicolumn{1}{|p{\maxVarWidth}|}{\centering .*} & \multicolumn{2}{p{\paraWidth}|}{must be a registered interpolator} \\\hline
\end{tabular*}

\vspace{0.5cm}\noindent {\bf [1]} \noindent \begin{verbatim}Lagrange polynomial interpolation (tensor product)\end{verbatim}\noindent {\bf [1]} \noindent \begin{verbatim}Lagrange polynomial interpolation (maximum degree)\end{verbatim}\noindent {\bf [1]} \noindent \begin{verbatim}Hermite polynomial interpolation\end{verbatim}\noindent \begin{tabular*}{\tableWidth}{|c|l@{\extracolsep{\fill}}r|}
\hline
\multicolumn{1}{|p{\maxVarWidth}}{hydro\_analysis\_interpolator\_options} & {\bf Scope:} private & STRING \\\hline
\multicolumn{3}{|p{\descWidth}|}{{\bf Description:}   {\em Options for the interpolator}} \\
\hline{\bf Range} & &  {\bf Default:} order=2 \\\multicolumn{1}{|p{\maxVarWidth}|}{\centering .*} & \multicolumn{2}{p{\paraWidth}|}{must be a valid option specification} \\\hline
\end{tabular*}

\vspace{0.5cm}\noindent \begin{tabular*}{\tableWidth}{|c|l@{\extracolsep{\fill}}r|}
\hline
\multicolumn{1}{|p{\maxVarWidth}}{hydro\_analysis\_r\_core} & {\bf Scope:} private & REAL \\\hline
\multicolumn{3}{|p{\descWidth}|}{{\bf Description:}   {\em size of region around densest point in which to compute the centroid of rho*x}} \\
\hline{\bf Range} & &  {\bf Default:} 4.0 \\\multicolumn{1}{|p{\maxVarWidth}|}{\centering 0:*} & \multicolumn{2}{p{\paraWidth}|}{any positive radius. suggested is 1/2 of stellar radius} \\\hline
\end{tabular*}

\vspace{0.5cm}\noindent \begin{tabular*}{\tableWidth}{|c|l@{\extracolsep{\fill}}r|}
\hline
\multicolumn{1}{|p{\maxVarWidth}}{hydro\_analysis\_rho\_max\_loc\_only\_positive\_x} & {\bf Scope:} private & BOOLEAN \\\hline
\multicolumn{3}{|p{\descWidth}|}{{\bf Description:}   {\em Restrict location search for density maximum to positive values of x}} \\
\hline & & {\bf Default:} false \\\hline
\end{tabular*}

\vspace{0.5cm}\noindent \begin{tabular*}{\tableWidth}{|c|l@{\extracolsep{\fill}}r|}
\hline
\multicolumn{1}{|p{\maxVarWidth}}{hydro\_analysis\_rho\_max\_loc\_use\_rotatingsymmetry180} & {\bf Scope:} private & BOOLEAN \\\hline
\multicolumn{3}{|p{\descWidth}|}{{\bf Description:}   {\em Map found maxima into positive x half-plane assuming pi-symmetry}} \\
\hline & & {\bf Default:} false \\\hline
\end{tabular*}

\vspace{0.5cm}\noindent \begin{tabular*}{\tableWidth}{|c|l@{\extracolsep{\fill}}r|}
\hline
\multicolumn{1}{|p{\maxVarWidth}}{hydro\_analysis\_rho\_max\_origin\_distance\_npoints} & {\bf Scope:} private & INT \\\hline
\multicolumn{3}{|p{\descWidth}|}{{\bf Description:}   {\em Number of points along the straight line for measuring proper distance}} \\
\hline{\bf Range} & &  {\bf Default:} 100 \\\multicolumn{1}{|p{\maxVarWidth}|}{\centering 1:*} & \multicolumn{2}{p{\paraWidth}|}{Any positive number} \\\hline
\end{tabular*}

\vspace{0.5cm}\noindent \begin{tabular*}{\tableWidth}{|c|l@{\extracolsep{\fill}}r|}
\hline
\multicolumn{1}{|p{\maxVarWidth}}{verbosity\_level} & {\bf Scope:} private & INT \\\hline
\multicolumn{3}{|p{\descWidth}|}{{\bf Description:}   {\em how much information to ouptut to the logs}} \\
\hline{\bf Range} & &  {\bf Default:} 1 \\\multicolumn{1}{|p{\maxVarWidth}|}{\centering } & \multicolumn{2}{p{\paraWidth}|}{Output nothing} \\\multicolumn{1}{|p{\maxVarWidth}|}{\centering 1} & \multicolumn{2}{p{\paraWidth}|}{warn when finding multiple maxima} \\\multicolumn{1}{|p{\maxVarWidth}|}{\centering 2} & \multicolumn{2}{p{\paraWidth}|}{also output location of maxima} \\\hline
\end{tabular*}

\vspace{0.5cm}\noindent \begin{tabular*}{\tableWidth}{|c|l@{\extracolsep{\fill}}r|}
\hline
\multicolumn{1}{|p{\maxVarWidth}}{restmass\_compute\_masses} & {\bf Scope:} restricted & BOOLEAN \\\hline
\multicolumn{3}{|p{\descWidth}|}{{\bf Description:}   {\em Should we compute the masses?}} \\
\hline & & {\bf Default:} no \\\hline
\end{tabular*}

\vspace{0.5cm}\noindent \begin{tabular*}{\tableWidth}{|c|l@{\extracolsep{\fill}}r|}
\hline
\multicolumn{1}{|p{\maxVarWidth}}{restmass\_masses\_nr} & {\bf Scope:} restricted & INT \\\hline
\multicolumn{3}{|p{\descWidth}|}{{\bf Description:}   {\em number of radii within which to compute the rest mass}} \\
\hline{\bf Range} & &  {\bf Default:} (none) \\\multicolumn{1}{|p{\maxVarWidth}|}{\centering 0:100} & \multicolumn{2}{p{\paraWidth}|}{Positive and hard-limited} \\\hline
\end{tabular*}

\vspace{0.5cm}\noindent \begin{tabular*}{\tableWidth}{|c|l@{\extracolsep{\fill}}r|}
\hline
\multicolumn{1}{|p{\maxVarWidth}}{restmass\_ref\_radius\_mass} & {\bf Scope:} restricted & REAL \\\hline
\multicolumn{3}{|p{\descWidth}|}{{\bf Description:}   {\em Radii within which mass will be computed}} \\
\hline{\bf Range} & &  {\bf Default:} 10.0 \\\multicolumn{1}{|p{\maxVarWidth}|}{\centering 0.0:} & \multicolumn{2}{p{\paraWidth}|}{Positive} \\\hline
\end{tabular*}

\vspace{0.5cm}\noindent \begin{tabular*}{\tableWidth}{|c|l@{\extracolsep{\fill}}r|}
\hline
\multicolumn{1}{|p{\maxVarWidth}}{restmass\_rho\_min} & {\bf Scope:} restricted & REAL \\\hline
\multicolumn{3}{|p{\descWidth}|}{{\bf Description:}   {\em Points with rest-mass density below this value are excluded (it could be set equal to the atmosphere value)}} \\
\hline{\bf Range} & &  {\bf Default:} 1.e-9 \\\multicolumn{1}{|p{\maxVarWidth}|}{\centering 0:} & \multicolumn{2}{p{\paraWidth}|}{} \\\hline
\end{tabular*}

\vspace{0.5cm}\noindent \begin{tabular*}{\tableWidth}{|c|l@{\extracolsep{\fill}}r|}
\hline
\multicolumn{1}{|p{\maxVarWidth}}{timelevels} & {\bf Scope:} shared from HYDROBASE & INT \\\hline
\end{tabular*}

\vspace{0.5cm}\parskip = 10pt 

\section{Interfaces} 


\parskip = 0pt

\vspace{3mm} \subsection*{General}

\noindent {\bf Implements}: 

hydro\_analysis
\vspace{2mm}

\noindent {\bf Inherits}: 

grid

hydrobase

admbase
\vspace{2mm}
\subsection*{Grid Variables}
\vspace{5mm}\subsubsection{PRIVATE GROUPS}

\vspace{5mm}

\begin{tabular*}{150mm}{|c|c@{\extracolsep{\fill}}|rl|} \hline 
~ {\bf Group Names} ~ & ~ {\bf Variable Names} ~  &{\bf Details} ~ & ~\\ 
\hline 
grid\_spacing\_product & grid\_spacing\_product & compact & 0 \\ 
 &  & description & product of cctk\_delta\_space \\ 
& ~ & description &  to be computed in local mode and later used in global mode \\ 
 &  & dimensions & 0 \\ 
 &  & distribution & CONSTANT \\ 
 &  & group type & SCALAR \\ 
 &  & tags & checkpoint="no" \\ 
 &  & timelevels & 1 \\ 
 &  & variable type & REAL \\ 
\hline 
hydro\_analysis\_masses & Hydro\_Analysis\_masses & compact & 0 \\ 
 &  & description & Baryonic masses at different radii \\ 
 &  & dimensions & 0 \\ 
 &  & distribution & CONSTANT \\ 
 &  & group type & SCALAR \\ 
 &  & tags & checkpoint="no" \\ 
 &  & timelevels & 1 \\ 
 &  & vararray\_size & restmass\_masses\_nr \\ 
 &  & variable type & REAL \\ 
\hline 
hydro\_analysis\_masses\_fractions & Hydro\_Analysis\_masses\_fractions & compact & 0 \\ 
 &  & description & Fractional Baryonic masses at different radii \\ 
 &  & dimensions & 0 \\ 
 &  & distribution & CONSTANT \\ 
 &  & group type & SCALAR \\ 
 &  & tags & checkpoint="no" \\ 
 &  & timelevels & 1 \\ 
 &  & vararray\_size & restmass\_masses\_nr \\ 
 &  & variable type & REAL \\ 
\hline 
hydro\_analysis\_total\_rest\_mass &  & compact & 0 \\ 
 & total\_rest\_mass & description & Total Baryonic mass \\ 
 &  & dimensions & 0 \\ 
 &  & distribution & CONSTANT \\ 
 &  & group type & SCALAR \\ 
 &  & tags & checkpoint="no" \\ 
 &  & timelevels & 1 \\ 
 &  & variable type & REAL \\ 
\hline 
hydro\_analysis\_masses\_temps & Hydro\_Analysis\_masses\_temps & compact & 0 \\ 
 &  & description & Temporaries for the mass calculation \\ 
 &  & dimensions & 3 \\ 
 &  & distribution & DEFAULT \\ 
 &  & group type & GF \\ 
 &  & tags & checkpoint="no" \\ 
 &  & timelevels & 3 \\ 
 &  & vararray\_size & restmass\_masses\_nr+1 \\ 
 &  & variable type & REAL \\ 
\hline 
\end{tabular*} 


\vspace{5mm}\subsubsection{PUBLIC GROUPS}

\vspace{5mm}

\begin{tabular*}{150mm}{|c|c@{\extracolsep{\fill}}|rl|} \hline 
~ {\bf Group Names} ~ & ~ {\bf Variable Names} ~  &{\bf Details} ~ & ~\\ 
\hline 
hydro\_analysis\_rho\_max & Hydro\_Analysis\_rho\_max & compact & 0 \\ 
 &  & description & value of the maximum of rho \\ 
 &  & dimensions & 0 \\ 
 &  & distribution & CONSTANT \\ 
 &  & group type & SCALAR \\ 
 &  & timelevels & 1 \\ 
 &  & variable type & REAL \\ 
\hline 
hydro\_analysis\_rho\_sum & Hydro\_Analysis\_rho\_sum & compact & 0 \\ 
 &  & description & value of the sum of rho \\ 
 &  & dimensions & 0 \\ 
 &  & distribution & CONSTANT \\ 
 &  & group type & SCALAR \\ 
 &  & timelevels & 1 \\ 
 &  & variable type & REAL \\ 
\hline 
hydro\_analysis\_rho\_max\_loc & Hydro\_Analysis\_rho\_max\_loc & compact & 0 \\ 
 &  & description & coordinate location of the maximum of rho \\ 
 &  & dimensions & 0 \\ 
 &  & distribution & CONSTANT \\ 
 &  & group type & SCALAR \\ 
 &  & timelevels & 1 \\ 
 &  & vararray\_size & 3 \\ 
 &  & variable type & REAL \\ 
\hline 
hydro\_analysis\_rho\_center\_volume\_weighted\_gf & Hydro\_Analysis\_rho\_center\_volume\_weighted\_gf & compact & 0 \\ 
 &  & description & temporary GF to obtain the coordinate location of the volume weighted center of mass via a reduction \\ 
 &  & dimensions & 3 \\ 
 &  & distribution & DEFAULT \\ 
 &  & group type & GF \\ 
 &  & tags & checkpoint="no" tensortypealias="u" \\ 
 &  & timelevels & 3 \\ 
 &  & vararray\_size & 3 \\ 
 &  & variable type & REAL \\ 
\hline 
hydro\_analysis\_rho\_center\_volume\_weighted & Hydro\_Analysis\_rho\_center\_volume\_weighted & compact & 0 \\ 
 &  & description & coordinate location of the volume weightes center of mass \\ 
 &  & dimensions & 0 \\ 
 &  & distribution & CONSTANT \\ 
 &  & group type & SCALAR \\ 
 &  & timelevels & 1 \\ 
 &  & vararray\_size & 3 \\ 
 &  & variable type & REAL \\ 
\hline 
hydro\_analysis\_rho\_max\_origin\_distance & Hydro\_Analysis\_rho\_max\_origin\_distance & compact & 0 \\ 
 &  & description & proper distance between the maximum of the density and the origin (along a straight coordinate line) \\ 
 &  & dimensions & 0 \\ 
 &  & distribution & CONSTANT \\ 
 &  & group type & SCALAR \\ 
 &  & timelevels & 1 \\ 
 &  & variable type & REAL \\ 
\hline 
\end{tabular*} 



\vspace{5mm}
\vspace{5mm}

\begin{tabular*}{150mm}{|c|c@{\extracolsep{\fill}}|rl|} \hline 
~ {\bf Group Names} ~ & ~ {\bf Variable Names} ~  &{\bf Details} ~ & ~ \\ 
\hline 
hydro\_analysis\_core\_rho\_sum & Hydro\_Analysis\_core\_rho\_sum & compact & 0 \\ 
 &  & description & value of the sum of rho in the core region \\ 
 &  & dimensions & 0 \\ 
 &  & distribution & CONSTANT \\ 
 &  & group type & SCALAR \\ 
 &  & timelevels & 1 \\ 
 &  & variable type & REAL \\ 
\hline 
hydro\_analysis\_core\_rho\_centroid\_gf & Hydro\_Analysis\_core\_rho\_centroid\_gf & compact & 0 \\ 
 &  & description & temporary GF to obtain the coordinate location of the centroid location of rho*x in the core region via a reduction \\ 
 &  & dimensions & 3 \\ 
 &  & distribution & DEFAULT \\ 
 &  & group type & GF \\ 
 &  & tags & checkpoint="no" tensortypealias="4u" \\ 
 &  & timelevels & 3 \\ 
 &  & vararray\_size & 4 \\ 
 &  & variable type & REAL \\ 
\hline 
hydro\_analysis\_core\_rho\_centroid & Hydro\_Analysis\_core\_rho\_centroid & compact & 0 \\ 
 &  & description & coordinate location of the centroid of rho*x in the core region \\ 
 &  & dimensions & 0 \\ 
 &  & distribution & CONSTANT \\ 
 &  & group type & SCALAR \\ 
 &  & timelevels & 1 \\ 
 &  & vararray\_size & 3 \\ 
 &  & variable type & REAL \\ 
\hline 
\end{tabular*} 



\vspace{5mm}\parskip = 10pt 

\section{Schedule} 


\parskip = 0pt


\noindent This section lists all the variables which are assigned storage by thorn EinsteinAnalysis/Hydro\_Analysis.  Storage can either last for the duration of the run ({\bf Always} means that if this thorn is activated storage will be assigned, {\bf Conditional} means that if this thorn is activated storage will be assigned for the duration of the run if some condition is met), or can be turned on for the duration of a schedule function.


\subsection*{Storage}

\hspace{5mm}

 \begin{tabular*}{160mm}{ll} 
~& {\bf Conditional:} \\ 
~ &  Hydro\_Analysis\_rho\_max\\ 
~ &  Hydro\_Analysis\_rho\_max\_loc\\ 
~ &  Hydro\_Analysis\_total\_rest\_mass\\ 
~ &  Hydro\_Analysis\_masses\_fractions\\ 
~ &  grid\_spacing\_product\\ 
~ &  Hydro\_Analysis\_rho\_sum\\ 
~ &  Hydro\_Analysis\_rho\_center\_volume\_weighted\\ 
~ &  Hydro\_Analysis\_rho\_center\_volume\_weighted\_gf[3]\\ 
~ &  Hydro\_Analysis\_rho\_max\_origin\_distance\\ 
~ &  Hydro\_Analysis\_core\_rho\_sum\\ 
~ &  Hydro\_Analysis\_core\_rho\_centroid\\ 
~ &  Hydro\_Analysis\_masses\_temps[3]\\ 
~ &  Hydro\_Analysis\_masses\\ 
~ & ~\\ 
\end{tabular*} 


\subsection*{Scheduled Functions}
\vspace{5mm}

\noindent {\bf CCTK\_BASEGRID}   (conditional) 

\hspace{5mm} hydro\_analysis\_init 

\hspace{5mm}{\it initialize variables } 


\hspace{5mm}

 \begin{tabular*}{160mm}{cll} 
~ & Language:  & c \\ 
~ & Options:  & global \\ 
~ & Type:  & function \\ 
\end{tabular*} 


\vspace{5mm}

\noindent {\bf CCTK\_POSTSTEP}   (conditional) 

\hspace{5mm} hydro\_analysis 

\hspace{5mm}{\it group for hydro\_analysis routines } 


\hspace{5mm}

 \begin{tabular*}{160mm}{cll} 
~ & Type:  & group \\ 
\end{tabular*} 


\vspace{5mm}

\noindent {\bf Hydro\_Analysis\_CompCoreRhoCentroid}   (conditional) 

\hspace{5mm} hydro\_analysis\_compcorerhocentroid\_getstorage 

\hspace{5mm}{\it get temporary storage for duration of reduction } 


\hspace{5mm}

 \begin{tabular*}{160mm}{cll} 
~ & Before:  & hydro\_analysis\_compcorerhocentroid\_preparereduction \\ 
~ & Language:  & c \\ 
~ & Options:  & global \\ 
~ & Type:  & function \\ 
\end{tabular*} 


\vspace{5mm}

\noindent {\bf Hydro\_Analysis\_CompCoreRhoCentroid}   (conditional) 

\hspace{5mm} hydro\_analysis\_compcorerhocentroid\_preparereduction 

\hspace{5mm}{\it prepare core center of mass densities for reduction } 


\hspace{5mm}

 \begin{tabular*}{160mm}{cll} 
~ & Language:  & c \\ 
~ & Options:  & global \\ 
~& ~ &loop-local\\ 
~ & Type:  & function \\ 
\end{tabular*} 


\vspace{5mm}

\noindent {\bf Hydro\_Analysis\_CompCoreRhoCentroid}   (conditional) 

\hspace{5mm} hydro\_analysis\_compcorerhocentroid\_reduction 

\hspace{5mm}{\it compute the global core center of mass reduction results } 


\hspace{5mm}

 \begin{tabular*}{160mm}{cll} 
~ & After:  & hydro\_analysis\_compcorerhocentroid\_preparereduction \\ 
~ & Language:  & c \\ 
~ & Options:  & global \\ 
~ & Type:  & function \\ 
\end{tabular*} 


\vspace{5mm}

\noindent {\bf Hydro\_Analysis\_CompCoreRhoCentroid}   (conditional) 

\hspace{5mm} hydro\_analysis\_compcorerhocentroid\_freestorage 

\hspace{5mm}{\it free temporary storage } 


\hspace{5mm}

 \begin{tabular*}{160mm}{cll} 
~ & After:  & hydro\_analysis\_compcorerhocentroid\_reduction \\ 
~ & Language:  & c \\ 
~ & Options:  & global \\ 
~ & Type:  & function \\ 
\end{tabular*} 


\vspace{5mm}

\noindent {\bf CCTK\_POSTSTEP}   (conditional) 

\hspace{5mm} hydro\_analysis\_masses 

\hspace{5mm}{\it computation of rest masses contained within given radii } 


\hspace{5mm}

 \begin{tabular*}{160mm}{cll} 
~ & Type:  & group \\ 
\end{tabular*} 


\vspace{5mm}

\noindent {\bf Hydro\_Analysis\_Masses}   (conditional) 

\hspace{5mm} hydro\_analysis\_masses\_local 

\hspace{5mm}{\it rest masses within given radii, local calculations } 


\hspace{5mm}

 \begin{tabular*}{160mm}{cll} 
~ & Language:  & fortran \\ 
~ & Type:  & function \\ 
\end{tabular*} 


\vspace{5mm}

\noindent {\bf Hydro\_Analysis\_Masses}   (conditional) 

\hspace{5mm} hydro\_analysis\_masses\_global 

\hspace{5mm}{\it rest masses within given radii, reduction operation } 


\hspace{5mm}

 \begin{tabular*}{160mm}{cll} 
~ & After:  & hydro\_analysis\_masses\_local \\ 
~ & Language:  & fortran \\ 
~ & Options:  & global \\ 
~ & Type:  & function \\ 
\end{tabular*} 


\vspace{5mm}

\noindent {\bf Hydro\_Analysis}   (conditional) 

\hspace{5mm} hydro\_analysis\_preparereduction 

\hspace{5mm}{\it compute the local reduction results } 


\hspace{5mm}

 \begin{tabular*}{160mm}{cll} 
~ & Language:  & c \\ 
~ & Options:  & global-early \\ 
~& ~ &loop-local\\ 
~ & Type:  & function \\ 
\end{tabular*} 


\vspace{5mm}

\noindent {\bf Hydro\_Analysis}   (conditional) 

\hspace{5mm} hydro\_analysis\_reduction 

\hspace{5mm}{\it compute the global reduction results } 


\hspace{5mm}

 \begin{tabular*}{160mm}{cll} 
~ & After:  & hydro\_analysis\_preparereduction \\ 
~ & Language:  & c \\ 
~ & Options:  & global \\ 
~ & Type:  & function \\ 
\end{tabular*} 


\vspace{5mm}

\noindent {\bf Hydro\_Analysis}   (conditional) 

\hspace{5mm} hydro\_analysis\_locationsearch 

\hspace{5mm}{\it look for the location of the maximum density } 


\hspace{5mm}

 \begin{tabular*}{160mm}{cll} 
~ & After:  & hydro\_analysis\_reduction \\ 
~ & Type:  & group \\ 
\end{tabular*} 


\vspace{5mm}

\noindent {\bf Hydro\_Analysis\_LocationSearch}   (conditional) 

\hspace{5mm} hydro\_analysis\_locationsearch\_setup 

\hspace{5mm}{\it prepare data structures for search } 


\hspace{5mm}

 \begin{tabular*}{160mm}{cll} 
~ & Language:  & c \\ 
~ & Options:  & global \\ 
~ & Type:  & function \\ 
\end{tabular*} 


\vspace{5mm}

\noindent {\bf Hydro\_Analysis\_LocationSearch}   (conditional) 

\hspace{5mm} hydro\_analysis\_locationsearch\_search 

\hspace{5mm}{\it search for the location of the maximum density } 


\hspace{5mm}

 \begin{tabular*}{160mm}{cll} 
~ & After:  & hydro\_analysis\_locationsearch\_setup \\ 
~ & Language:  & c \\ 
~ & Options:  & global \\ 
~& ~ &loop-local\\ 
~ & Type:  & function \\ 
\end{tabular*} 


\vspace{5mm}

\noindent {\bf Hydro\_Analysis\_LocationSearch}   (conditional) 

\hspace{5mm} hydro\_analysis\_locationsearch\_combine 

\hspace{5mm}{\it communicate and verify the location of the maximum density } 


\hspace{5mm}

 \begin{tabular*}{160mm}{cll} 
~ & After:  & hydro\_analysis\_locationsearch\_search \\ 
~ & Language:  & c \\ 
~ & Options:  & global \\ 
~ & Type:  & function \\ 
\end{tabular*} 


\vspace{5mm}

\noindent {\bf Hydro\_Analysis}   (conditional) 

\hspace{5mm} hydro\_analysis\_findseparation 

\hspace{5mm}{\it compute the proper distance between the maximum of the density and the origin (along a straight coordinate line) } 


\hspace{5mm}

 \begin{tabular*}{160mm}{cll} 
~ & After:  & hydro\_analysis\_locationsearch \\ 
~ & Language:  & fortran \\ 
~ & Options:  & global \\ 
~ & Type:  & function \\ 
\end{tabular*} 


\vspace{5mm}

\noindent {\bf Hydro\_Analysis}   (conditional) 

\hspace{5mm} hydro\_analysis\_compcorerhocentroid 

\hspace{5mm}{\it compute center of mass of core region } 


\hspace{5mm}

 \begin{tabular*}{160mm}{cll} 
~ & After:  & hydro\_analysis\_locationsearch \\ 
~ & Type:  & group \\ 
\end{tabular*} 



\vspace{5mm}\parskip = 10pt 
\end{document}
