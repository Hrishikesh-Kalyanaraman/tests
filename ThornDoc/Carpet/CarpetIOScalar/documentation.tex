% *======================================================================*
%  Cactus Thorn template for ThornGuide documentation
%  Author: Ian Kelley
%  Date: Sun Jun 02, 2002
%
%  Thorn documentation in the latex file doc/documentation.tex
%  will be included in ThornGuides built with the Cactus make system.
%  The scripts employed by the make system automatically include
%  pages about variables, parameters and scheduling parsed from the
%  relevant thorn CCL files.
%
%  This template contains guidelines which help to assure that your
%  documentation will be correctly added to ThornGuides. More
%  information is available in the Cactus UsersGuide.
%
%  Guidelines:
%   - Do not change anything before the line
%       % START CACTUS THORNGUIDE",
%     except for filling in the title, author, date, etc. fields.
%        - Each of these fields should only be on ONE line.
%        - Author names should be separated with a \\ or a comma.
%   - You can define your own macros, but they must appear after
%     the START CACTUS THORNGUIDE line, and must not redefine standard
%     latex commands.
%   - To avoid name clashes with other thorns, 'labels', 'citations',
%     'references', and 'image' names should conform to the following
%     convention:
%       ARRANGEMENT_THORN_LABEL
%     For example, an image wave.eps in the arrangement CactusWave and
%     thorn WaveToyC should be renamed to CactusWave_WaveToyC_wave.eps
%   - Graphics should only be included using the graphicx package.
%     More specifically, with the "\includegraphics" command.  Do
%     not specify any graphic file extensions in your .tex file. This
%     will allow us to create a PDF version of the ThornGuide
%     via pdflatex.
%   - References should be included with the latex "\bibitem" command.
%   - Use \begin{abstract}...\end{abstract} instead of \abstract{...}
%   - Do not use \appendix, instead include any appendices you need as
%     standard sections.
%   - For the benefit of our Perl scripts, and for future extensions,
%     please use simple latex.
%
% *======================================================================*
%
% Example of including a graphic image:
%    \begin{figure}[ht]
% 	\begin{center}
%    	   \includegraphics[width=6cm]{/home/runner/work/tests/tests/arrangements/Carpet/CarpetIOScalar/doc/MyArrangement_MyThorn_MyFigure}
% 	\end{center}
% 	\caption{Illustration of this and that}
% 	\label{MyArrangement_MyThorn_MyLabel}
%    \end{figure}
%
% Example of using a label:
%   \label{MyArrangement_MyThorn_MyLabel}
%
% Example of a citation:
%    \cite{MyArrangement_MyThorn_Author99}
%
% Example of including a reference
%   \bibitem{MyArrangement_MyThorn_Author99}
%   {J. Author, {\em The Title of the Book, Journal, or periodical}, 1 (1999),
%   1--16. {\tt http://www.nowhere.com/}}
%
% *======================================================================*

\documentclass{article}

% Use the Cactus ThornGuide style file
% (Automatically used from Cactus distribution, if you have a
%  thorn without the Cactus Flesh download this from the Cactus
%  homepage at www.cactuscode.org)
\usepackage{../../../../../doc/latex/cactus}

\newlength{\tableWidth} \newlength{\maxVarWidth} \newlength{\paraWidth} \newlength{\descWidth} \begin{document}

% The author of the documentation
\author{Erik Schnetter \textless schnetter@aei.mpg.de\textgreater}

% The title of the document (not necessarily the name of the Thorn)
\title{CapetIOScalar}

% the date your document was last changed, if your document is in CVS,
% please use:
\date{June 11, 2004}

\maketitle

% Do not delete next line
% START CACTUS THORNGUIDE

% Add all definitions used in this documentation here
%   \def\mydef etc

% Add an abstract for this thorn's documentation
\begin{abstract}

\end{abstract}

% The following sections are suggestive only.
% Remove them or add your own.

\section{Introduction}



%%%%%%%%%%%%%%%%%%%%%%%%%%%%%%%%%%%%%%%%%%%%%%%%%%%%%%%%%%%%%%%%%%%%%%%%%%%%
\section{Utilities}
\label{sec:utils}

These scripts may be useful handling CarpetIOASCII data:
\begin{itemize}
\item {\bf Carpet2ygraph.pl} A perl script to convert to a format
  suitable for the {\it xgraph} or {\it ygraph} packages of P. Walker
  and D. Pollney.
\item {\bf mergeCarpetIOScalar.pl} A perl script to remove duplicate datasets from file(s).
\end{itemize}

The first script resides in CarpetIOASCII/src/utils and was meant primarily to convert 1D data,
since the CarpetIOScalar data format is easily readable also by ygraph. Anyway, this script converts
also scalar data files to the original 2-column ygraph format.
Usage: {\tt Carpet2ygraphCat.pl Inputfile}.

The second script can be used to merge CarpetIOScalar output written before and after recovery. It
reads one or more files in CarpetIOScalar format and writes their contents to STDOUT, eliminating
duplicate datasets (all but the last occurance are discarded), which may be created when a run died
abruptly without checkpointing at the last output itaration and is recovered from a previous
checkpoint.  
Usage: {\tt mergeCarpetIOScalar.pl list of files}. \\
Example 1: {\tt mergeCarpetIOScalar.pl alp.norm1.asc > alp.norm1.asc.merged}.\\ 
Example 2: {\tt mergeCarpetIOScalar.pl alp.norm1.asc.firstCHKPT alp.norm1.asc.secondCHKPT > alp.norm1.asc.merged}.\\
This script was written by Thomas Radke.


\begin{thebibliography}{9}

\end{thebibliography}

% Do not delete next line
% END CACTUS THORNGUIDE



\section{Parameters} 


\parskip = 0pt

\setlength{\tableWidth}{160mm}

\setlength{\paraWidth}{\tableWidth}
\setlength{\descWidth}{\tableWidth}
\settowidth{\maxVarWidth}{all\_reductions\_in\_one\_file}

\addtolength{\paraWidth}{-\maxVarWidth}
\addtolength{\paraWidth}{-\columnsep}
\addtolength{\paraWidth}{-\columnsep}
\addtolength{\paraWidth}{-\columnsep}

\addtolength{\descWidth}{-\columnsep}
\addtolength{\descWidth}{-\columnsep}
\addtolength{\descWidth}{-\columnsep}
\noindent \begin{tabular*}{\tableWidth}{|c|l@{\extracolsep{\fill}}r|}
\hline
\multicolumn{1}{|p{\maxVarWidth}}{all\_reductions\_in\_one\_file} & {\bf Scope:} private & BOOLEAN \\\hline
\multicolumn{3}{|p{\descWidth}|}{{\bf Description:}   {\em Write all requested reductions in one file instead of per reduction}} \\
\hline & & {\bf Default:} no \\\hline
\end{tabular*}

\vspace{0.5cm}\noindent \begin{tabular*}{\tableWidth}{|c|l@{\extracolsep{\fill}}r|}
\hline
\multicolumn{1}{|p{\maxVarWidth}}{one\_file\_per\_group} & {\bf Scope:} private & BOOLEAN \\\hline
\multicolumn{3}{|p{\descWidth}|}{{\bf Description:}   {\em Write one file per group instead of per variable}} \\
\hline & & {\bf Default:} no \\\hline
\end{tabular*}

\vspace{0.5cm}\noindent \begin{tabular*}{\tableWidth}{|c|l@{\extracolsep{\fill}}r|}
\hline
\multicolumn{1}{|p{\maxVarWidth}}{out\_precision} & {\bf Scope:} private & INT \\\hline
\multicolumn{3}{|p{\descWidth}|}{{\bf Description:}   {\em How many digits to output floating-point numbers with}} \\
\hline{\bf Range} & &  {\bf Default:} 15 \\\multicolumn{1}{|p{\maxVarWidth}|}{\centering 0:*} & \multicolumn{2}{p{\paraWidth}|}{Number of precision digits} \\\hline
\end{tabular*}

\vspace{0.5cm}\noindent \begin{tabular*}{\tableWidth}{|c|l@{\extracolsep{\fill}}r|}
\hline
\multicolumn{1}{|p{\maxVarWidth}}{outscalar\_criterion} & {\bf Scope:} private & KEYWORD \\\hline
\multicolumn{3}{|p{\descWidth}|}{{\bf Description:}   {\em Criterion to select scalar output intervals, overrides out\_every}} \\
\hline{\bf Range} & &  {\bf Default:} iteration \\\multicolumn{1}{|p{\maxVarWidth}|}{\centering never} & \multicolumn{2}{p{\paraWidth}|}{Never output} \\\multicolumn{1}{|p{\maxVarWidth}|}{\centering iteration} & \multicolumn{2}{p{\paraWidth}|}{Output every so many iterations} \\\multicolumn{1}{|p{\maxVarWidth}|}{\centering divisor} & \multicolumn{2}{p{\paraWidth}|}{Output if iteration mod divisor == 0.} \\\multicolumn{1}{|p{\maxVarWidth}|}{\centering time} & \multicolumn{2}{p{\paraWidth}|}{Output every that much coordinate time} \\\hline
\end{tabular*}

\vspace{0.5cm}\noindent \begin{tabular*}{\tableWidth}{|c|l@{\extracolsep{\fill}}r|}
\hline
\multicolumn{1}{|p{\maxVarWidth}}{outscalar\_dir} & {\bf Scope:} private & STRING \\\hline
\multicolumn{3}{|p{\descWidth}|}{{\bf Description:}   {\em Name of scalar output directory, overrides out\_dir}} \\
\hline{\bf Range} & &  {\bf Default:} (none) \\\multicolumn{1}{|p{\maxVarWidth}|}{\centering \^\$} & \multicolumn{2}{p{\paraWidth}|}{Empty: use IO::out\_dir} \\\multicolumn{1}{|p{\maxVarWidth}|}{\centering .+} & \multicolumn{2}{p{\paraWidth}|}{Not empty: directory name} \\\hline
\end{tabular*}

\vspace{0.5cm}\noindent \begin{tabular*}{\tableWidth}{|c|l@{\extracolsep{\fill}}r|}
\hline
\multicolumn{1}{|p{\maxVarWidth}}{outscalar\_dt} & {\bf Scope:} private & REAL \\\hline
\multicolumn{3}{|p{\descWidth}|}{{\bf Description:}   {\em How often to do scalar output, overrides out\_dt}} \\
\hline{\bf Range} & &  {\bf Default:} -2 \\\multicolumn{1}{|p{\maxVarWidth}|}{\centering (0:*} & \multicolumn{2}{p{\paraWidth}|}{In intervals of that much coordinate time} \\\multicolumn{1}{|p{\maxVarWidth}|}{\centering } & \multicolumn{2}{p{\paraWidth}|}{As often as possible} \\\multicolumn{1}{|p{\maxVarWidth}|}{\centering -1} & \multicolumn{2}{p{\paraWidth}|}{Disable output} \\\multicolumn{1}{|p{\maxVarWidth}|}{\centering -2} & \multicolumn{2}{p{\paraWidth}|}{Default to IO::out\_dt} \\\hline
\end{tabular*}

\vspace{0.5cm}\noindent \begin{tabular*}{\tableWidth}{|c|l@{\extracolsep{\fill}}r|}
\hline
\multicolumn{1}{|p{\maxVarWidth}}{outscalar\_every} & {\bf Scope:} private & INT \\\hline
\multicolumn{3}{|p{\descWidth}|}{{\bf Description:}   {\em How often to do scalar output, overrides IO::out\_every}} \\
\hline{\bf Range} & &  {\bf Default:} -2 \\\multicolumn{1}{|p{\maxVarWidth}|}{\centering 1:*} & \multicolumn{2}{p{\paraWidth}|}{Output every so many time steps} \\\multicolumn{1}{|p{\maxVarWidth}|}{\centering -1:0} & \multicolumn{2}{p{\paraWidth}|}{No output} \\\multicolumn{1}{|p{\maxVarWidth}|}{\centering -2} & \multicolumn{2}{p{\paraWidth}|}{Default to IO::out\_every} \\\hline
\end{tabular*}

\vspace{0.5cm}\noindent \begin{tabular*}{\tableWidth}{|c|l@{\extracolsep{\fill}}r|}
\hline
\multicolumn{1}{|p{\maxVarWidth}}{outscalar\_reductions} & {\bf Scope:} private & STRING \\\hline
\multicolumn{3}{|p{\descWidth}|}{{\bf Description:}   {\em List of reductions to output in scalar form}} \\
\hline{\bf Range} & &  {\bf Default:} count minimum maximum sum average norm1 norm2 norm\_inf icount iaverage inorm1 inorm2 \\\multicolumn{1}{|p{\maxVarWidth}|}{\centering } & \multicolumn{2}{p{\paraWidth}|}{A regex which matches everything} \\\hline
\end{tabular*}

\vspace{0.5cm}\noindent \begin{tabular*}{\tableWidth}{|c|l@{\extracolsep{\fill}}r|}
\hline
\multicolumn{1}{|p{\maxVarWidth}}{outscalar\_vars} & {\bf Scope:} private & STRING \\\hline
\multicolumn{3}{|p{\descWidth}|}{{\bf Description:}   {\em Variables to output in scalar form}} \\
\hline{\bf Range} & &  {\bf Default:} (none) \\\multicolumn{1}{|p{\maxVarWidth}|}{\centering } & \multicolumn{2}{p{\paraWidth}|}{A regex which matches everything} \\\hline
\end{tabular*}

\vspace{0.5cm}\noindent \begin{tabular*}{\tableWidth}{|c|l@{\extracolsep{\fill}}r|}
\hline
\multicolumn{1}{|p{\maxVarWidth}}{out\_criterion} & {\bf Scope:} shared from IO & KEYWORD \\\hline
\end{tabular*}

\vspace{0.5cm}\noindent \begin{tabular*}{\tableWidth}{|c|l@{\extracolsep{\fill}}r|}
\hline
\multicolumn{1}{|p{\maxVarWidth}}{out\_dir} & {\bf Scope:} shared from IO & STRING \\\hline
\end{tabular*}

\vspace{0.5cm}\noindent \begin{tabular*}{\tableWidth}{|c|l@{\extracolsep{\fill}}r|}
\hline
\multicolumn{1}{|p{\maxVarWidth}}{out\_dt} & {\bf Scope:} shared from IO & REAL \\\hline
\end{tabular*}

\vspace{0.5cm}\noindent \begin{tabular*}{\tableWidth}{|c|l@{\extracolsep{\fill}}r|}
\hline
\multicolumn{1}{|p{\maxVarWidth}}{out\_every} & {\bf Scope:} shared from IO & INT \\\hline
\end{tabular*}

\vspace{0.5cm}\noindent \begin{tabular*}{\tableWidth}{|c|l@{\extracolsep{\fill}}r|}
\hline
\multicolumn{1}{|p{\maxVarWidth}}{out\_fileinfo} & {\bf Scope:} shared from IO & KEYWORD \\\hline
\end{tabular*}

\vspace{0.5cm}\noindent \begin{tabular*}{\tableWidth}{|c|l@{\extracolsep{\fill}}r|}
\hline
\multicolumn{1}{|p{\maxVarWidth}}{out\_group\_separator} & {\bf Scope:} shared from IO & STRING \\\hline
\end{tabular*}

\vspace{0.5cm}\noindent \begin{tabular*}{\tableWidth}{|c|l@{\extracolsep{\fill}}r|}
\hline
\multicolumn{1}{|p{\maxVarWidth}}{strict\_io\_parameter\_check} & {\bf Scope:} shared from IO & BOOLEAN \\\hline
\end{tabular*}

\vspace{0.5cm}\noindent \begin{tabular*}{\tableWidth}{|c|l@{\extracolsep{\fill}}r|}
\hline
\multicolumn{1}{|p{\maxVarWidth}}{verbose} & {\bf Scope:} shared from IO & KEYWORD \\\hline
\end{tabular*}

\vspace{0.5cm}\parskip = 10pt 

\section{Interfaces} 


\parskip = 0pt

\vspace{3mm} \subsection*{General}

\noindent {\bf Implements}: 

ioscalar
\vspace{2mm}
\subsection*{Grid Variables}
\vspace{5mm}\subsubsection{PRIVATE GROUPS}

\vspace{5mm}

\begin{tabular*}{150mm}{|c|c@{\extracolsep{\fill}}|rl|} \hline 
~ {\bf Group Names} ~ & ~ {\bf Variable Names} ~  &{\bf Details} ~ & ~\\ 
\hline 
last\_output\_iteration & last\_output\_iteration & compact & 0 \\ 
 &  & dimensions & 0 \\ 
 &  & distribution & CONSTANT \\ 
 &  & group type & SCALAR \\ 
 &  & timelevels & 1 \\ 
 &  & variable type & INT \\ 
\hline 
last\_output\_time & last\_output\_time & compact & 0 \\ 
 &  & dimensions & 0 \\ 
 &  & distribution & CONSTANT \\ 
 &  & group type & SCALAR \\ 
 &  & timelevels & 1 \\ 
 &  & variable type & REAL \\ 
\hline 
this\_iteration & this\_iteration & compact & 0 \\ 
 &  & dimensions & 0 \\ 
 &  & distribution & CONSTANT \\ 
 &  & group type & SCALAR \\ 
 &  & timelevels & 1 \\ 
 &  & variable type & INT \\ 
\hline 
\end{tabular*} 



\vspace{5mm}

\noindent {\bf Uses header}: 

Timer.hh

carpet.hh

typecase.hh

typeprops.hh
\vspace{2mm}\parskip = 10pt 

\section{Schedule} 


\parskip = 0pt


\noindent This section lists all the variables which are assigned storage by thorn Carpet/CarpetIOScalar.  Storage can either last for the duration of the run ({\bf Always} means that if this thorn is activated storage will be assigned, {\bf Conditional} means that if this thorn is activated storage will be assigned for the duration of the run if some condition is met), or can be turned on for the duration of a schedule function.


\subsection*{Storage}

\hspace{5mm}

 \begin{tabular*}{160mm}{ll} 

{\bf Always:}&  ~ \\ 
 last\_output\_iteration last\_output\_time this\_iteration & ~\\ 
~ & ~\\ 
\end{tabular*} 


\subsection*{Scheduled Functions}
\vspace{5mm}

\noindent {\bf CCTK\_STARTUP} 

\hspace{5mm} carpetioscalarstartup 

\hspace{5mm}{\it startup routine } 


\hspace{5mm}

 \begin{tabular*}{160mm}{cll} 
~ & After:  & ioutil\_startup \\ 
~ & Language:  & c \\ 
~ & Options:  & global \\ 
~ & Type:  & function \\ 
\end{tabular*} 


\vspace{5mm}

\noindent {\bf CCTK\_INITIAL} 

\hspace{5mm} carpetioscalarinit 

\hspace{5mm}{\it initialisation routine } 


\hspace{5mm}

 \begin{tabular*}{160mm}{cll} 
~ & Language:  & c \\ 
~ & Options:  & global \\ 
~ & Type:  & function \\ 
~ & Writes:  & carpetioscalar::last\_output\_iteration \\ 
~& ~ &carpetioscalar::last\_output\_time\\ 
~& ~ &carpetioscalar::this\_iteration\\ 
\end{tabular*} 



\vspace{5mm}\parskip = 10pt 
\end{document}
