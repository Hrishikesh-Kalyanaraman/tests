
\section{Schedule} 


\parskip = 0pt


\noindent This section lists all the variables which are assigned storage by thorn EinsteinInitialData/ReadInterpolate.  Storage can either last for the duration of the run ({\bf Always} means that if this thorn is activated storage will be assigned, {\bf Conditional} means that if this thorn is activated storage will be assigned for the duration of the run if some condition is met), or can be turned on for the duration of a schedule function.


\subsection*{Storage}

\hspace{5mm}

 \begin{tabular*}{160mm}{ll} 
~& {\bf Conditional:} \\ 
~ &  test\_values[3] test\_results\\ 
~ & ~\\ 
\end{tabular*} 


\subsection*{Scheduled Functions}
\vspace{5mm}

\noindent {\bf CCTK\_PARAMCHECK}   (conditional) 

\hspace{5mm} readinterpolate\_paramcheck 

\hspace{5mm}{\it sanity check given parameters } 


\hspace{5mm}

 \begin{tabular*}{160mm}{cll} 
~ & Language:  & c \\ 
~ & Options:  & global \\ 
~ & Type:  & function \\ 
\end{tabular*} 


\vspace{5mm}

\noindent {\bf CCTK\_INITIAL}   (conditional) 

\hspace{5mm} readinterpolate\_readdata 

\hspace{5mm}{\it read in datasets from disk } 


\hspace{5mm}

 \begin{tabular*}{160mm}{cll} 
~ & After:  & admbase\_initialdata \\ 
~& ~ &hydrobase\_initial\\ 
~ & Before:  & admbase\_postinitial \\ 
~& ~ &hydrobase\_prim2coninitial\\ 
~ & Type:  & group \\ 
\end{tabular*} 


\vspace{5mm}

\noindent {\bf ReadInterpolate\_ReadData}   (conditional) 

\hspace{5mm} readinterpolate\_read 

\hspace{5mm}{\it read in datasets } 


\hspace{5mm}

 \begin{tabular*}{160mm}{cll} 
~ & Language:  & c \\ 
~ & Options:  & level \\ 
~ & Storage:  & reflevelseen \\ 
~& ~ &interpthispoint\\ 
~& ~ &interp\_coords\\ 
~ & Type:  & function \\ 
\end{tabular*} 


\vspace{5mm}

\noindent {\bf CCTK\_POSTPOSTINITIAL}   (conditional) 

\hspace{5mm} readinterpolate\_freecache 

\hspace{5mm}{\it free memory used for dataset caches } 


\hspace{5mm}

 \begin{tabular*}{160mm}{cll} 
~ & Language:  & c \\ 
~ & Options:  & global \\ 
~ & Type:  & function \\ 
\end{tabular*} 


\vspace{5mm}

\noindent {\bf ReadInterpolate\_ReadData}   (conditional) 

\hspace{5mm} readinterpolate\_enforcesymmetry 

\hspace{5mm}{\it enforce symmeries if desired } 


\hspace{5mm}

 \begin{tabular*}{160mm}{cll} 
~ & After:  & readinterpolate\_read \\ 
~ & Language:  & c \\ 
~ & Options:  & level \\ 
~ & Type:  & function \\ 
\end{tabular*} 


\vspace{5mm}

\noindent {\bf ReadInterpolate\_ReadData}   (conditional) 

\hspace{5mm} applybcs 

\hspace{5mm}{\it apply symmetry conditions to read in variables } 


\hspace{5mm}

 \begin{tabular*}{160mm}{cll} 
~ & After:  & readinterpolate\_enforcesymmetry \\ 
~ & Type:  & group \\ 
\end{tabular*} 


\vspace{5mm}

\noindent {\bf CCTK\_INITIAL}   (conditional) 

\hspace{5mm} readinterpolate\_generatetestdata 

\hspace{5mm}{\it generate polynomial test data } 


\hspace{5mm}

 \begin{tabular*}{160mm}{cll} 
~ & Language:  & c \\ 
~ & Type:  & function \\ 
\end{tabular*} 


\vspace{5mm}

\noindent {\bf CCTK\_POSTPOSTINITIAL}   (conditional) 

\hspace{5mm} readinterpolate\_comparetestdata 

\hspace{5mm}{\it compare to polynomial test data } 


\hspace{5mm}

 \begin{tabular*}{160mm}{cll} 
~ & Language:  & c \\ 
~ & Type:  & function \\ 
\end{tabular*} 


\subsection*{Aliased Functions}

\hspace{5mm}

 \begin{tabular*}{160mm}{ll} 

{\bf Alias Name:} ~~~~~~~ & {\bf Function Name:} \\ 
ApplyBCs & ReadInterpolate\_ApplyBCs \\ 
\end{tabular*} 



\vspace{5mm}\parskip = 10pt 
